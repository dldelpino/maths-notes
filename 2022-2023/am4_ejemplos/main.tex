\documentclass[12pt]{report}

\usepackage{graphicx}
\usepackage[a4paper, total={7in, 10in}]{geometry}
\usepackage[utf8]{inputenc}
\usepackage[spanish]{babel}
\usepackage{amsmath,amsfonts,amssymb,amsthm}
\usepackage{hyperref} % Para poder insertar hiperenlaces a secciones del documento
\usepackage[table, x11names]{xcolor} % Para cambiar las letras de color
\usepackage{dirtytalk} % Para usar comillas de apertura y cierre
\usepackage{fbox} % Para las cajas en las demostraciones de "si y solo si" y doble contención
\usepackage{mathtools}
\usepackage{multicol} % Para dividir una lista en varias columnas
\usepackage{soul} % Para cambiar de línea con palabras subrayadas
\usepackage{imakeidx}
\usepackage{graphicx}
\usepackage{faktor} % Conjuntos cociente
\usepackage{float} % Para que algunas figuras no se coloquen al inicio de la página
\usepackage{centernot} % Para negar símbolos como \implies
\usepackage{soulutf8} % Para pegar el subrayado al texto
\usepackage{esvect}
\usepackage{spalign}
\usepackage{array}
\usepackage{enumitem}

\makeindex[columns=3, intoc]

\setlength{\columnsep}{0.8cm} % Recta que separa dos columnas (multicols)
\setlength{\columnseprule}{1.5pt}

\makeatletter % Para que el título de los teoremas estén en negrita
\def\th@plain{%
  \thm@notefont{}
  \itshape
}
\def\th@definition{
  \thm@notefont{}
  \normalfont
}
\makeatother

\graphicspath{{./images/}}

\newtheorem{proposition}{Proposición}[chapter]
\newtheorem{corollary}{Corolario}[chapter]
\newtheorem{theorem}{Teorema}[chapter]
\theoremstyle{definition}
\newtheorem{definition}{Definición}[chapter]
\theoremstyle{definition}
\newtheorem{example}{Ejemplo}[chapter]
\theoremstyle{remark}
\newtheorem*{obs}{Observación} % * para que no se numeren
\renewcommand*{\proofname}{Demostración}
\setcounter{chapter}{-1} % Para que el primer tema sea el Tema 0 y no el Tema 1
\addto\captionsspanish{\renewcommand{\chaptername}{Tema}} % Para que ponga "Tema 1" en vez de "Capítulo 1"
\addto\captionsspanish{\renewcommand{\contentsname}{Índice}} % Para cambiar el título del índice

\setuldepth{Berlin}

\newcommand*{\Scale}[2][4]{\scalebox{#1}{$#2$}}%

% Shortcuts:
\newcommand{\R}{\mathbb R}
\newcommand{\N}{\mathbb N}
\newcommand{\Z}{\mathbb Z}
\newcommand{\Q}{\mathbb Q}

\begin{document}

\small

% Longitud antes y después de una expresión matemática:
\setlength{\abovedisplayskip}{10pt}
\setlength{\belowdisplayskip}{10pt}

\begin{center}
    \textbf{Ejemplos importantes del Tema 1}
\end{center}

\begin{enumerate}[leftmargin=*]
    \item \textit{Sucesión de funciones continuas que converge puntualmente a una función no continua.}
    \[
    \begin{aligned}[t]
    f_k \colon [0,1] &\longrightarrow \R \\
    x &\longmapsto f_k(x)=x^k
    \end{aligned}
    \]
    También sirve como ejemplo de función que converge puntualmente pero no uniformemente.
    \item \textit{Sucesión de funciones derivables que converge puntual y uniformemente a una función no derivable.}
    \[
    \begin{aligned}[t]
    f_k \colon [0,1] &\longrightarrow \R \\
    x &\longmapsto f_k(x)=\sqrt{x^2+\frac{1}{k}}
    \end{aligned}
    \]
    \item \textit{Sucesión de funciones integrables que converge puntualmemente a una función no integrable.}
    \[
    \begin{aligned}[t]
    f_k \colon [0,1] &\longrightarrow \R \\
    x &\longmapsto f_k(x)=\begin{cases}
        1 $\ \ si \ \ $ x=x_1, x=x_2,\mathellipsis, x=x_{k-1} $\ ó \ $ x = x_k \\
        0 $\ \ en caso contrario$
    \end{cases}
    \end{aligned}
    \]
    donde $\{x_k\}$ es una enumeración de los racionales de $[0,1]$.
    \item \textit{Sucesiones uniformemente convergentes cuyo producto no converge uniformemente.}
    \[
    \begin{aligned}[t]
    f_k \colon \R &\longrightarrow \R \\
    x &\longmapsto f_k(x)=x+\frac{1}{k}
    \end{aligned} \qquad \qquad    \begin{aligned}[t]
    g_k \colon \R &\longrightarrow \R \\
    x &\longmapsto g_k(x)=x+\frac{1}{k}
    \end{aligned}
    \]
    \item \textit{Serie funcional de una sucesión de funciones uniformemente convergente a la función nula que no converge uniformemente}.
    \[\sum_{n=1}^\infty \frac{x^n}{n} \ \textup{en} \ [0,1]\]
    \item \textit{Serie funcional que converge uniformemente pero no verifica las hipótesis del criterio de Weierstrass}.
    \[\sum_{n=1}^\infty \frac{(1-x)x^n}{\log(n+1)} \ \textup{en} \ [0,1]\]
    También sirve cualquier serie funcional que no converja absolutamente pero sí uniformemente, como
    \[\sum_{n=1}^\infty (-1)^n \frac{1}{x+n} \ \textup{en} \ (0,\infty)\]
    \item \textit{Función de clase infinito y no desarrollable en serie de potencias.}
    \[
    \begin{aligned}[t]
    f \colon \R &\longrightarrow \R \\
    x &\longmapsto f(x)=\begin{cases}
        e^{-\frac{1}{x^2}} $\ \ si $ x\neq 0 \\
        0 \phantom{^{-\frac{1}{x^2}}} $\ \ si $ x = 0
    \end{cases}
    \end{aligned}
    \]
    No es desarrollable en serie de potencias en $a = 0$.
\end{enumerate}

\thispagestyle{empty}

\pagebreak

\thispagestyle{empty}

\begin{center}
    \textbf{Ejemplos importantes del Tema 2}
\end{center}

\begin{enumerate}[leftmargin=*]
    \item \textit{Conjunto acotado y de medida cero que no tiene volumen cero.}
    \[\Q \cap [0,1]\]
    \item \textit{Función integrable con secciones no integrables.}
    \[
    \begin{aligned}[t]
    f \colon [0,1] \times [0,1] &\longrightarrow \R \\
    (x,y) &\longmapsto f(x,y) = \begin{cases}
        1 \phantom{-\frac{1}{q}} \ \ \ \ $si$ \ x \in \R \setminus \Q \\
        1 \phantom{-\frac{1}{q}} \ \ \ \ $si$ \ x \in \Q, y \in \R \setminus \Q \\
        1-\frac{1}{q} \ \ \: $si$ \ x, y \in \Q \ $y$ \ x = \frac{p}{q} \ $es frac. irred.$
    \end{cases}
    \end{aligned}
    \]
    Las secciones no integrables son las $f_x$.
    \item \textit{Función cuyas integrales iteradas existen y son distintas.}
    \[
    \begin{aligned}[t]
        f \colon [0,1] \times [0,1] &\longrightarrow \R \\
        (x,y) &\longmapsto f(x,y) = \begin{cases}
            \frac{x-y}{(x+y)^3} \ \ $si$ \ x>0,y>0 \\
            0 \ \ \ \ \ \ \ \: \: $en caso contrario$
        \end{cases}
    \end{aligned}
    \]
    El problema es que $f$ no es integrable, pues no es acotada.
    \item \textit{Conjunto acotado que no es J-medible.}
    \[A = \bigcup_{i=1}^\infty \, (x_i-r_i,x_i+r_i)\]
    donde $\{x_i\}_{i=1}^\infty$ es una enumeración de los racionales de $(0,1)$ y $r_i$ es tal que $(x_i-r_i,x_i+r_i) \subset (0,1)$ y $r_i < \frac{1}{2^{i+2}}$. La frontera es $[0,1] \setminus A$, que se prueba que no tiene medida cero usando que $\sum_{i=1}^\infty \frac{1}{2^{i+2}}<1$.
\end{enumerate}

\pagebreak

\thispagestyle{empty}

\begin{center}
    \textbf{Ejemplos importantes del Tema 3}
\end{center}

\begin{enumerate}[leftmargin=*]
    \item \textit{Campo vectorial no conservativo cuyas derivadas cruzadas coinciden.}
    \[\begin{aligned}[t]
    F \colon \R^2 \setminus \{(0,0)\} &\longrightarrow \R^2 \\
    (x,y) &\longmapsto F(x,y) = \biggl(\frac{y}{x^2+y^2},-\frac{x}{x^2+y^2}\biggr)
    \end{aligned}\]
    Se tiene que $\frac{\partial F_1}{\partial y} = \frac{\partial F_2}{\partial x}$ pero el campo no es conservativo porque la integral sobre la circunferencia de centro $(0,0)$ y radio $1$ no es nula.
\end{enumerate}
\end{document}

