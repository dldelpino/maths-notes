% Actualizado a día 2023-02-23

\documentclass[12pt]{report}

\usepackage{graphicx}
\usepackage[a4paper, total={6.5in, 9in}, top=1.5in]{geometry}
\usepackage[utf8]{inputenc}
\usepackage[spanish]{babel}
\decimalpoint
\usepackage{amsmath,amsfonts,amssymb,amsthm}
\usepackage{fancyhdr}

\parindent=0pt

% Shortcuts:
\newcommand{\R}{\mathbb R}
\newcommand{\N}{\mathbb N}
\newcommand{\Z}{\mathbb Z}
\newcommand{\Q}{\mathbb Q}

\pagestyle{fancy}
\fancyhead[L]{\textbf{Métodos Numéricos I}}
\fancyhead[R]{\textbf{Examen final - 2017/02/06}}

\begin{document}
\textbf{1. } Sea $x \in [0,1]$ y sea $\beta \in \N, \beta > 1$. Dividimos el intervalo $[0,1]$ en intervalos de longitud $\frac{1}{\beta}$:
\[[0,1] = [0,\frac{1}{\beta}] \cup [\frac{1}{\beta},\frac{2}{\beta}] \cup \mathellipsis \cup [\frac{\beta-1}{\beta},1]\]
Tomamos $b_1 \in \N, 1 \leq b_1 < \beta$ tal que
\[\frac{b_1}{\beta} \leq x < \frac{b_1}{\beta}+\frac{1}{\beta}\] Ahora dividimos el intervalo $[\frac{b_1}{\beta},\frac{b_1}{\beta}+1]$ en intervalos de longitud $\frac{1}{\beta^2}$, y tomamos $b_2 \in \N$ con $ 1 \leq b_2 < \beta$ y tal que
\[\frac{b_1}{\beta}+\frac{b_2}{\beta^2} \leq x < \frac{b_1}{\beta}+\frac{b_2}{\beta^2} + \frac{1}{\beta^2}\]
Repitiendo este proceso para cada $n \in \N$, encontramos $b_n \in \N$ con $1 \leq \beta_n < \beta$ tal que
\[\frac{b_1}{\beta}+\frac{b_2}{\beta^2}+\mathellipsis+\frac{b_n}{\beta^n} \leq x < \frac{b_1}{\beta}+\frac{b_2}{\beta^2}+\mathellipsis+\frac{b_n}{\beta^n} + \frac{1}{\beta^n}\]
La sucesión $\{S_n\} = \{\frac{b_1}{\beta}+\frac{b_2}{\beta^2}+\mathellipsis+\frac{b_n}{\beta^n}$ verifica

\begin{equation}
0 \leq x - S_n < \frac{1}{\beta^n} \ \forall \ n \in \N
\end{equation}

Y por la regla del sándwich, tenemos que $\lim{\{S_n\}} = x$. Por tanto, podemos escribir
\[x = \beta^{-1}b_1 + \beta^{-2}b_2 + \mathellipsis + \beta^{-n}b_n + \mathellipsis = (0'b_1b_2 \mathellipsis b_n \mathellipsis)_{\beta}\]
que es la expresión de $x$ en base $\beta$.

\vspace{2mm}
Por otra parte, el truncamiento de $x \in [0,1]$ a $n$ dígitos fraccionarios en base $\beta$ es precisamente $tr_n(x) = S_n = \frac{b_1}{\beta}+\frac{b_2}{\beta^2}+\mathellipsis+\frac{b_n}{\beta^n}$, así que, como consecuencia de (1), podemos dar la cota de error
\[|x - tr_n(x)| \leq \beta^{-n}\]

\vspace{2mm}
\textbf{2. } En primer lugar, vemos que la sucesión $\{x_{n+1}\} = \{g(x_n)\}$ está bien definida para cualquier elección de semilla, ya que, por hipótesis, $g(x) \in J \ \forall \ x \in J$. Sea $l$ el único punto fijo de $g$ en $J = [a,b]$ y sea $C \in [0,1)$ la constante de contractividad de $g$. Veamos, por inducción, que para todo $n \in \N$ se cumple
\[|x_{n} - l| \leq C^n(b-a)\]
\begin{itemize}
    \item Para $n = 1$, usando la contractividad de $g$, se tiene que 
    \[|x_1 - l| = |g(x_0) - g(l)| \leq C|x_0 -l| \leq C(b-a)\]
    \item Supongamos que para cierto $n \in \N$ se cumple $|x_n - l| \leq C^n(b-a)$ y veamos que $|x_{n+1}-l| \leq C^{n+1}(b-a)$. Como $g$ es contractiva,
    \[|x_{n+1} - l| = |g(x_n) - g(l)| \leq C|x_n - l|  \overset{{\textrm{HI}}}{\leq} C \cdot C^n(b-a) = C^{n+1}(b-a)\]
    Por tanto, como $0 \leq C < 1$, se tiene que $\lim{\{C^n\}} = 0$, y por la regla del sándwich, $\lim{\{|x_{n+1} - l|\}} = 0$, o lo que es lo mismo, $\lim{\{g(x_n)\}} = l$.
\end{itemize}

\vspace{2mm}
\textbf{3. } Tenemos que
\begin{itemize}
    \item $1 = 0 \cdot 7 + 1 \implies c_0 = 0, r_0 = 1$
    \item $2 \cdot 1 = 0 \cdot 7 + 2 \implies c_1 = 0, r_1 = 2$
    \item $2 \cdot 2 = 0 \cdot 7 + 4 \implies c_2 = 0, r_2 = 4$
    \item $2 \cdot 4 = 1 \cdot 7 + 1 \implies c_3 = 1, r_3 = 1$
    \item $2 \cdot 1 = 0 \cdot 7 + 2 \implies c_4 = 0, r_4 = 2$
\end{itemize}
A partir de aquí, vamos a empezar a obtener dígitos repetidos. Por tanto,
\[\frac{1}{7} = (0'\overline{001})_2\]
así que el truncamiento a 6 dígitos sería $tr_6(\frac{1}{7}) = (0'001001)_2$. La expresión en base 10 sería
\[(0'001001)_2 = 1 \cdot 2^{-3} + 1 \cdot 2^{-6} = \frac{1}{8} + \frac{1}{64} = 0'125 + 0.125^2 = 0.125 + 0.015625 = 0'140625\]
mientras que el error cometido es
\[|\frac{1}{7} - \frac{9}{64}| = |\frac{64 - 63}{448}| = \frac{1}{448} \leq \frac{1}{64} = 2^{-6}\]
Por tanto, la aproximación satisface la cota del Ejercicio 1.

\vspace{2mm}
\textbf{4. }

\vspace{2mm}
\textbf{(a)} Sea $f \colon \R \to \R, f(x) = x^3-x+\frac{1}{3\sqrt{3}}$. Tenemos que $f'(x) = 3x^2 - 1$, luego 
\[f'(x) = 0 \iff x =\pm\frac{1}{\sqrt{3}}\]
Por el teorema de Rolle, la ecuación tiene, a lo sumo, 3 soluciones: una en $(-\infty,-\frac{1}{\sqrt{3}}]$, otra en $[-\frac{1}{\sqrt{3}},\frac{1}{\sqrt{3}}]$ y otra en $[\frac{1}{\sqrt{3}},\infty)$. Además,
\begin{itemize}
    \item $\displaystyle \lim_{x \to -\infty}f(x) = - \infty$
    \item $f(-\frac{1}{\sqrt{3}}) = \frac{1}{\sqrt{3}} > 0$
    \item $f(\frac{1}{\sqrt{3}}) = \frac{2}{3\sqrt{3}} - \frac{1}{\sqrt{3}} < 0$
    \item $\displaystyle \lim_{x \to \infty}f(x) = \infty$
\end{itemize}
Por el teorema de Bolzano, la ecuación tiene una solución $l_1$ en $(-\infty,-\frac{1}{\sqrt{3}}]$, otra solución $l_2$ en $[-\frac{1}{\sqrt{3}},\frac{1}{\sqrt{3}}]$ y otra solución $l_3$ en $[\frac{1}{\sqrt{3}},\infty)$. Por tanto, la ecuación tiene exactamente 3 soluciones.

\vspace{2mm}
\textbf{(b)} La función de iteración del método es $g(x) = x^3 + \frac{1}{3\sqrt{3}}$, que verifica
\[|g'(x)| = |3x^2| = 3x^2\]
Por tanto,
\begin{itemize}
    \item $|g'(x)| = 1 \iff x = \pm\frac{1}{\sqrt{3}}$
    \item $|g'(x)| > 1 \iff x \in (-\infty, -\frac{1}{\sqrt{3}}) \cup (\frac{1}{\sqrt{3}},\infty)$
    \item $|g'(x)| < 1 \iff x \in (-\frac{1}{\sqrt{3}},\frac{1}{\sqrt{3}})$
\end{itemize}
Como hemos visto anteriormente que $\frac{1}{\sqrt{3}}$ y $-\frac{1}{\sqrt{3}}$ no son soluciones de la ecuación, entonces podemos asegurar que $|g'(l_1)| > 1, |g'(l_2)| < 1$ y $|g'(l_3)| > 1$. Esto significa que el método no converge localmente (y por tanto no converge) a $l_1$ ni a $l_3$, mientras que converge localmente a $l_2$.

\vspace{2mm}
\textbf{(c)} Tomamos el intervalo $J = [-\frac{1}{3},\frac{1}{3}] \subset (-\frac{1}{\sqrt{3}},\frac{1}{\sqrt{3}})$, que verifica
\begin{itemize}
    \item $l_2 \in J$, pues $f(-\frac{1}{3}) = -\frac{1}{9} + \frac{1}{3} + \frac{1}{3\sqrt{3}} > 0$ y $f(\frac{1}{3}) = \frac{1}{9} - \frac{1}{3} + \frac{1}{3\sqrt{3}} < 0$.
    \item Las ecuaciones $f(x) = 0$ y $g(x) = x$ son equivalentes en $J$, pues
    \[f(x) = 0 \iff x^3 - x + \frac{1}{3\sqrt{3}} = 0 \iff x^3 + \frac{1}{3\sqrt{3}} = x \iff g(x) = x\]
    \item $g(J) \subset J$, pues $g$ es creciente en $J$ (ya que $g'(x) = 3x^2 \geq 0 \ \forall \ x \in J$) y por tanto se tiene que
    \[g([-\frac{1}{3},\frac{1}{3}]) = [g(-\frac{1}{3}),g(\frac{1}{3})] = [-\frac{1}{9} + \frac{1}{3\sqrt{3}},\frac{1}{9} + \frac{1}{3\sqrt{3}}]\]
    que está contenido en $J$ porque
    \begin{itemize}
        \item $\displaystyle -\frac{1}{3} < 0 < -\frac{1}{9}+\frac{1}{3\sqrt{3}} < \frac{1}{3\sqrt{3}} < \frac{1}{3}$
        \item $\displaystyle -\frac{1}{3} < -\frac{1}{9} + \frac{1}{3\sqrt{3}} < \frac{1}{9} + \frac{1}{3\sqrt{3}} = \frac{1}{3}(\frac{1}{3} + \frac{1}{\sqrt{3}}) < \frac{1}{3}$
    \end{itemize}
    \item $g$ es contractiva en $J$, pues $g''(x) = 6x = 0 \iff x = 0$, y como
    \begin{itemize}
        \item $g'(-\frac{1}{3}) = \frac{1}{3}$
        \item $g'(\frac{1}{3}) = \frac{1}{3}$
        \item $g'(0) = 0$
    \end{itemize}
    entonces $\frac{1}{3}$ y $-\frac{1}{3}$ son máximos absoutos de $g'$ (y por tanto de $|g'|$). Esto significa que
    \[|g'(x)| \leq \frac{1}{3} = C < 1 \ \forall \ x \in J\]
\end{itemize}
Por tanto, el método converge hacia la única raíz de $f$ en $J$.

\vspace{2mm}
\textbf{(d)}
La menor raíz positiva es $l_2$, ya que 
\begin{itemize}
    \item $f(\frac{1}{6}) = \frac{1}{216} - \frac{1}{6} + \frac{1}{3\sqrt{3}} > 0'003 - 0'17 + 0'19 > 0$
    \item $f(\frac{1}{3}) = \frac{1}{9} - \frac{1}{3} + \frac{1}{3\sqrt{3}} < 0'12 -0.3+0.2 < 0$
\end{itemize}
Por tanto, $l_2 \in [\frac{1}{6},\frac{1}{3}]$. Tomando la semilla $x_0 = \frac{1}{6}$, se tiene que
\begin{itemize}
    \item $f(\frac{1}{6})f(\frac{1}{3}) < 0$
    \item $f'(x) = 3x^2 - 1 \neq 0 \ \forall \ x \in [\frac{1}{6},\frac{1}{3}]$ ya que 
    \[\frac{1}{6} \leq x \leq \frac{1}{3} \implies \frac{\sqrt{3}}{6} \leq \sqrt{3}x \leq \frac{1}{\sqrt{3}} \implies \frac{1}{12} \leq 3x^2 \leq \frac{1}{3} < 1 \implies f'(x) = 3x^2 - 1 < 0\]
    \item $f''(x) = 6x \neq 0 \ \forall \ x \in [\frac{1}{6},\frac{1}{3}]$
    \item $f''(x_0)f(x_0) = 1 \cdot f(\frac{1}{6}) \geq 0$
\end{itemize}
Con esta semilla, tenemos asegurada la convergencia del método de Newton hacia la raíz $l_2$.

\vspace{2mm}
\textbf{5. }

\vspace{2mm}
\textbf{(a)} Las diferencias divididas son
\begin{itemize}
    \item $f[x_0] = 1; f[x_1] = -1; f[x_2] = 1; f[x_3] = -1; f[x_4] = 1$
    \item $f[x_0,x_1] = -2; f[x_1,x_2] = 2; f[x_2,x_3] = -2; f[x_3,x_4] = 2$
    \item $f[x_0,x_1,x_2] = 2; f[x_1,x_2,x_3] = -2; f[x_2,x_3,x_4] = 2$
    \item $f[x_0,x_1,x_2,x_3] = -\frac{4}{3}; f[x_1,x_2,x_3,x_4] = \frac{4}{3}$
    \item $f[x_0,x_1,x_2,x_3,x_4] = \frac{2}{3}$
\end{itemize}
Por tanto,
$p(x) = 1 -2(x+2) +2(x+1)(x+2) -\frac{4}{3}x(x+1)(x+2) + \frac{2}{3}x(x-1)(x+1)(x+2)$. Se verifica la cota uniforme
\[|f(x) - p(x)| \leq \frac{M_5}{5!}|x(x+2)(x+1)(x-1)(x-2)| \leq \frac{M_5}{5!}(2-(-2))^5 = \frac{\pi^5 \cdot 4^5}{120}\]
ya que $\displaystyle M_5 = \max_{x \in [-2,2]} |f^{(5)}(x)| = \max_{x \in [-2,2]} |-\pi^5\sin(x)| = \pi^5$ 

\textbf{(b)} Vamos a calcular el polinomio $p_1$ de grado menor o igual que $2$ que interpola los puntos $(-2,f(-2)),(-1,f(-1)),(0,f(0))$ y el polinomio $p_2$ de grado menor o igual que $2$ que interpola los puntos $(0,f(0)),(1,f(1)),(2,f(2))$.
\begin{itemize}
    \item Para calcular $p_1$, hallamos las diferencias divididas:
    \begin{itemize}
        \item $f[x_0] = 1; f[x_1] = -1; f[x_2] = 1$
        \item $f[x_0,x_1] = -2; f[x_1,x_2] = 2$
        \item $f[x_0,x_1,x_2] = 2$
    \end{itemize}
    Por tanto, $p_1(x) = 1 -2(x+2) + 2(x+1)(x+2)$
    \item Para calcular $p_2$, hallamos las diferencias divididas:
    \begin{itemize}
        \item $f[x_0] = 1; f[x_1] = -1; f[x_2] = 1$
        \item $f[x_0,x_1] = -2; f[x_1,x_2] = 2$
        \item $f[x_0,x_1,x_2] = 2$
    \end{itemize}
    Por tanto, $p_2(x) = 1 -2x + 2x(x-1)$
\end{itemize}
El polinomio $\tilde{p}$ es 
\[\tilde{p}(x) = 
\begin{cases}
1 -2(x+2) + 2(x+1)(x+2) & $si$ \ x \in [-2,0] \\
1 -2x + 2x(x-1) & $si$ \ x \in [0,2]
\end{cases}\]
Para hallar una cota uniforme del error, usamos la cota óptima al interpolar los extremos de un intervalo y su punto medio en los polinomios $p_1$ y $p_2$. Si $x \in [-2,0]$,
\[|f(x) - \tilde{p}(x)| = |f(x) - p_1(x)| \leq \frac{M_3}{72\sqrt{3}}2^3 = \frac{\pi^3 \cdot 8}{72 \sqrt{3}} = \frac{\pi^3}{9\sqrt{3}}\]
donde $\displaystyle M_3 = \max_{x \in [-2,0]} |f'''(x)| = \pi^3$. Si $x \in [0,2]$, se obtiene la misma cota de error. Por tanto, dicha cota es uniforme en el intervalo $[-2,2]$.

\vspace{2mm}
\textbf{(c)} El valor exacto de la integral es
\[I(f) = \int_{-2}^2 \cos{(\pi x)}dx = \Biggl[ \frac{1}{\pi}\sin{(\pi x)} \Biggr]_{-2}^2 = 0\]
Tenemos que
\[
\begin{aligned}[t]
I(\tilde{p}) = \int_{-2}^2 \tilde{p}(x)dx &= \int_{-2}^0 p_1(x)dx + \int_0^2 p_2(x)dx = \int_{-2}^0 (2x^2 +4x+1)dx + \int_0^2 (2x^2-4x+1)dx = \\
&= \Biggl[ \frac{2}{3}x^3 + 2x^2 + x\Biggr]_{-2}^0 + \Biggl[ \frac{2}{3}x^3 - 2x^2 + x\Biggr]_{0}^2 = \frac{16}{3}-8+2 + \frac{16}{3}-8+2 = -\frac{4}{3}
\end{aligned}
\]

Por tanto, el error es
\[|I(f) - I(\tilde{p})| = \frac{4}{3}\]
Por otra parte,
\[I^S(f) = \frac{4}{6}(f(-2) + 4f(0) + f(2)) = \frac{2}{3}(1+4+2) = \frac{14}{3}\]
En este caso, el error que se comete es
\[|I(f) - I^S(f)| = \frac{14}{3}\]
Vemos que la primera aproximación es mejor que la segunda.

\vspace{2mm}
\textbf{(d) } Si $N \in \N$ es par, entonces...
\begin{itemize}
    \item $f[-N] = 1$
    \item $f[-N,-N+1] = -2$
    \item $f[-N,-N+1,-N+2] = \frac{4}{2} = 2$
    \item $f[-N,-N+1,-N+2,-N+3] = -\frac{4}{3}$
    \item $f[-N,-N+1,-N+2,-N+3,-N+4] = \frac{\frac{8}{3}}{4} = \frac{8}{12}$
    \item $f[-N,-N+1,-N+2,-N+3,-N+4,-N+5] = \frac{-\frac{16}{12}}{5} = -\frac{16}{60}$
    \item $f[-N,-N+1,-N+2,-N+3,-N+4,-N+5,-N+6] = \frac{\frac{32}{60}}{6} = \frac{32}{360}$
\end{itemize}
En general, $f[-N,-N+1,\mathellipsis,-N+i] = (-1)^i\frac{2^{i-1}}{\frac{i!}{2}} = (-1)^i\frac{2^i}{i!} \ \forall \ i = 0,\mathellipsis,2N$, luego
\[p(x) = \sum_{i = 0}^{2N} (-1)^i\frac{2^i}{i!}(x+N)(x+N-1) \mathellipsis (x+N-i+1)\]
Si $N$ es impar, basta cambiar $(-1)^i$ por $(-1)^{i+1}$.
\end{document}