\documentclass[12pt]{report}

\usepackage{graphicx}
\usepackage[a4paper, total={6.5in, 9in}]{geometry}
\usepackage[utf8]{inputenc}
\usepackage[spanish]{babel}
\decimalpoint
\usepackage{amsmath,amsfonts,amssymb,amsthm}
\usepackage{fancyhdr}
\usepackage{multicol}
\usepackage{fbox}

\parindent=0pt

% Shortcuts:
\newcommand{\R}{\mathbb R}
\newcommand{\N}{\mathbb N}
\newcommand{\Z}{\mathbb Z}
\newcommand{\Q}{\mathbb Q}

\begin{document}

\small

\textbf{Geometría Diferencial de Curvas y Superficies} \hfill \textbf{Examen de junio de 2021}
\linebreak

\textbf{1.} \textit{Sea $\alpha \colon (0,\pi) \to \R^2$ la curva dada por $\alpha(t) = (\sen t, \cos t +\log(\tan\frac{t}{2}))$.}
\begin{itemize}
    \item[\textit{(a)}] \textit{Estúdiese la regularidad de $\alpha$.}
    \item[\textit{(b)}] \textit{Pruébese que en cualquier punto de la traza de $\alpha$, la longitud del segmento de la recta tangente entre dicho punto y el eje  $y$ es constante e igual a uno.}
    \item[\textit{(c)}] \textit{Calcúlese la curvatura de $\alpha$.}
\end{itemize}

\vspace{2mm}
\textbf{(a) } En primer lugar, $\alpha$ está bien definida porque $\frac{t}{2} \in (0,\frac{\pi}{2}) \subset (-\frac{\pi}{2},\frac{\pi}{2})$ para todo $t \in (0,\pi)$, y además $\tan\frac{t}{2} > 0$ para todo $t \in (0,\pi)$. Hay que estudiar la existencia de puntos singulares de $\alpha$. Se tiene que
\[\alpha'(t) = 0 \iff \biggl(\cos t, -\sen t+\frac{1}{2\tan\frac{t}{2}\cos^2\frac{t}{2}}\biggr) \iff \biggl(\cos t, \frac{\cos t}{\tan t}\biggr) = 0\]
Se deduce que $t = \frac{\pi}{2}$ es el único punto singular de $\alpha$.

\vspace{2mm}
\textbf{(b) } La recta tangente en un punto $\alpha(t)$ de la traza de $\alpha$ es aquella que pasa por $\alpha(t)$ y tiene vector director $\alpha'(t)$. Hay que hallar entonces el punto en que dicha recta corta al eje $y$. La ecuación punto-pediente de la recta tangente es

\[
y = \frac{1}{\tan t}(x - \sen t)+\cos t + \log (\tan \textstyle \frac{t}{2})
\]
Poniendo $x = 0$ se obtiene $y = \log(\tan\frac{t}{2})$, luego el punto de corte de la recta tangente con el eje $y$ es $(0,\log(\tan\frac{t}{2}))$. La longitud $L$ pedida no es más que la distancia entre $\alpha(t)$ y este punto:
\[L = \sqrt{\sen^2t+(\cos t +\log (\textstyle \tan \frac{t}{2})-\log(\tan\frac{t}{2}))^2} = 1\]

\vspace{2mm}
\textbf{(c)} Como $\alpha$ no está parametrizada por el arco, la curvatura será
\[k_\alpha(t)=\frac{\langle \alpha''(t),J\alpha'(t)\rangle}{||\alpha'(t)||^3}\]
y como lo que queda consiste en hacer cuentas, el ejercicio se da por terminado.

\vspace{4mm}
\textbf{2. } \textit{Sea $f \colon U \to \R$ una función diferenciable en un abierto $U \subset \R^2$. Pruébese que
\[\textup{gr} \, f = \{(x,y,z) \in \R^3 \colon (x,y) \in U, z = f(x,y)\}\]
es una superficie regular.}

\vspace{2mm}
Va a probarse que $(U,\varphi)$ es una carta global de $S$, donde
\[
\begin{aligned}[t]
\varphi \colon U &\longrightarrow \R^3 \\
(u,v) &\longmapsto \varphi(u,v) = (u,v,f(u,v))
\end{aligned}
\]
Como $f$ es diferenciable, entonces $\varphi$ también. La inyectividad es clara, pues si $(u,v),(u',v') \in U$ son tales que $(u,v,f(u,v)) = (u',v',f(u'v'))$, entonces tiene que ser $u = u'$ y $v = v'$, de forma que $(u,v) = (u',v')$. La inversa de $\varphi \colon U \to \varphi(U)$ es
\[
\begin{aligned}[t]
\psi \colon \varphi(U) &\longrightarrow U \\
(u,v,w) &\longmapsto \psi(u,v,w) = (u,v)
\end{aligned}
\]
como se comprueba fácilmente. Evidentemente, es una aplicación continua, así que $\varphi \colon U \to \varphi(U)$ es homeomorfismo. Por último, la matriz jacobiana de $\varphi$ en cada punto $q = (u,v) \in U$ es
\[J\varphi_q = \begin{pmatrix}
    1 & 0 \\
    0 & 1 \\
    f_u & f_v
\end{pmatrix}\]
que tiene siempre rango máximo. Por tanto $d\varphi_q$ es inyectiva para cada $q \in U$. Esto prueba que $(U,\varphi)$ es una carta global de $S$ y en consecuencia $S$ es una superficie regular.

\vspace{4mm}
\textbf{3. } \textit{Sea 
\[S = \{(x,y,z) \in \R^3 \colon x^2+y^2=1, z>0\}\] el semicilindro vertical, y sea 
\[C = \{(x,y,z) \in \R^3 \colon x^2+y^2=z^2,z>0\}
\] el cono. Considérese la aplicación $f \colon S \to C$ definida por $f(x,y,z) = (zx,zy,z)$.}

\begin{itemize}
    \item[\textit{(a)}] \textit{Calcúlese la curvatura de Gauss y la curvatura media de $S$ y $C$.}
    \item[\textit{(b)}] \textit{Compruébese que $f$ está bien definida y es un difeomorfismo.}
    \item[\textit{(c)}] \textit{¿Es $f$ una isometría?}
\end{itemize}

\vspace{2mm}
\textbf{(a) } El semicilindro vertical $S$ queda recubierto por las cartas $(U,\varphi)$ y $(V, \psi)$, siendo los dominios de las cartas $U = (0,2\pi) \times (0,\infty)$ y $V = (-\pi, \pi) \times (0,\infty)$, mientras que 
\[
\begin{aligned}[t]
    \varphi \colon U &\longrightarrow \R^3 \\
    (u,v) &\longmapsto \varphi(u,v) = (\cos u, \sen u, v)
\end{aligned} \qquad
\begin{aligned}[t]
    \psi \colon V &\longrightarrow \R^3 \\
    (u,v) &\longmapsto \psi(u,v) = (\cos u, \sen u, v)
\end{aligned}
\]
Se tiene que
\[\varphi_u = (-\sen u, \cos u, 0) \qquad \varphi_v = (0,0,1) \qquad \psi_u = (-\sen u, \cos u, 0) \qquad \psi_v = (0,0,1)\]
Como se van a obtener los mismos resultados con cada una de las cartas, de aquí en adelante se trabajará solo con la primera de ellas. Los coeficientes de la métrica son
\[E = 1 \qquad F = 0 \qquad G = 1\]
Por otro lado,
\[\varphi_{uu} = (-\cos u, -\sen u, 0) \qquad \varphi_{vv} = (0,0,0) \qquad \varphi_{uv} = (0,0,0)\]
luego
\[e = -1 \qquad f = 0 \qquad g = 0\]
Las curvaturas del semicilindro vertical serían
\[k_S = \frac{eg-f^2}{EG-F^2} = 0 \qquad H_S = \frac{1}{2}\frac{eG-2fF+gE}{EG-F^2} = -\frac{1}{2}\]
En cuanto al cono, puede ser recubierto por las cartas $(U,\varphi)$, $(V,\psi)$ siendo los dominios de las cartas los mismos de antes, mientras que
\[
\begin{aligned}[t]
    \varphi \colon U &\longrightarrow \R^3 \\
    (u,v) &\longmapsto \varphi(u,v) = (v\cos u, v\sen u, v)
\end{aligned} \qquad
\begin{aligned}[t]
    \psi \colon V &\longrightarrow \R^3 \\
    (u,v) &\longmapsto \psi(u,v) = (v \cos u, v \sen u, v)
\end{aligned}
\]
Como se van a obtener los mismos resultados con cada una de las cartas, de aquí en adelante se trabajará solo con la primera de ellas. Se tiene que
\[\varphi_u = (-v\sen u, v\cos u, 0) \qquad \varphi_v = (\cos u, \sen u, 1)\]
Los coeficientes de la métrica son
\[E = v^2 \qquad F = 0 \qquad G = 2\]
Por otro lado,
\[\varphi_{uu} = (-v\cos u, -v\sen u, 0) \qquad \varphi_{vv} = (0,0,0) \qquad \varphi_{uv} = (-\sen u, \cos u, 0)\]
luego
\[e = -\frac{v^2}{\sqrt{2}v} = -\frac{v}{\sqrt{2}} \qquad f = 0 \qquad g = 0\]
Las curvaturas del cono serían
\[k_C = \frac{eg-f^2}{EG-F^2} = 0 \qquad H_C = \frac{1}{2}\frac{eG-2fF+gE}{EG-F^2} = -\frac{v}{2\sqrt{2}v^2} = -\frac{1}{2\sqrt{2}v}\]

\textbf{(b) } Si $(x,y,z) \in S$ y $f(x,y,z) = (zx,zy,z) = (x',y',z')$, entonces
\[x'^2+y'^2 = z^2x^2+z^2y^2 = z^2 = z'^2\]
luego $(x',y',z') = f(x,y,z) \in C$ y por tanto $f$ está bien definida. Es fácil comprobar (queriendo decir con esto que no va a hacerse) que $f$ es difeomorfismo.

\vspace{2mm}
\textbf{(c) } Dado $p = (p_1,p_2,p_3) \in S$, hay que estudiar si se tiene
\[\langle df_p(v),df_p(w) \rangle = \langle v,w \rangle\]
para todos $v,w\in T_pS$. Primero se calculará $df_p$. Sea $v = (v_1,v_2,v_3) \in T_pS$ representado por la curva $\alpha = (x,y,z)$. Entonces
\[
\begin{aligned}[t]
df_p(v) &= (f \circ \alpha)'(0) \\
&= (z(t)x(t),z(t)y(t),z(t))'(0) \\
&= (z'(0)x(0)+z(0)x'(0),z'(0)y(0)+z(0)y'(0),z'(0)) \\
&= (v_3p_1+p_3v_1,v_3p_2+p_3v_2,v_3)
\end{aligned}
\]
y por tanto,
\[\langle df_p(v),df_p(v) \rangle = (v_3p_1+p_3v_1)^2+(v_3p_2+p_3v_2)^2+v_3^2\]
que evidentemente no coincide con $\langle v,v \rangle$. Como $df_p$ no preserva la norma, entonces tampoco preserva el producto escalar, así que puede concluirse que $f$ no es una isometría.

\end{document}
