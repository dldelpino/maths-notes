% Actualizado a día 2023-02-23

\documentclass[12pt]{report}

\usepackage{graphicx}
\usepackage[a4paper, total={6.5in, 9in}, top=1.5in]{geometry}
\usepackage[utf8]{inputenc}
\usepackage[spanish]{babel}
\decimalpoint
\usepackage{amsmath,amsfonts,amssymb,amsthm}
\usepackage{fancyhdr}

\parindent=0pt

% Shortcuts:
\newcommand{\R}{\mathbb R}
\newcommand{\N}{\mathbb N}
\newcommand{\Z}{\mathbb Z}
\newcommand{\Q}{\mathbb Q}

\pagestyle{fancy}
\fancyhead[L]{\textbf{Métodos Numéricos I}}
\fancyhead[R]{\textbf{Examen final - 2020/02/06}}

\begin{document}
\textbf{1. } Tenemos que
\begin{itemize}
    \item $1 = 0 \cdot 7 + 1 \implies c_0 = 0, r_0 = 1$
    \item $2 \cdot 1 = 0 \cdot 7 + 2 \implies c_1 = 0, r_1 = 2$
    \item $2 \cdot 2 = 0 \cdot 7 + 4 \implies c_2 = 0, r_2 = 4$
    \item $2 \cdot 4 = 1 \cdot 7 + 1 \implies c_3 = 1, r_3 = 1$
\end{itemize}
A partir de aquí, empezamos a obtener cocientes repetidos, luego
\[\frac{1}{7} = (0'001001001 \mathellipsis)_2 = (0'\overline{001})_2\]
Como la cifra $b_8$ es cero, entonces el redondeo en base 2 es $rd_7(\frac{1}{7}) = (0'001001)_2$. Por otra parte,
\[\frac{1}{7} = 7^{-1} = (0'1)_7\]
así que el redondeo en base 7 es $rd_7(\frac{1}{7}) = (0'1)_7$. En el primer caso, el error cometido es
\[\biggl| \frac{1}{7} - (0'001001)_2 \biggr| = \biggl| \frac{1}{7} - 2^{-3} - 2^{-6} \biggr| = \biggl| \frac{1}{7} - \frac{1}{8} - \frac{1}{64} \biggr| = \biggl| \frac{512-448-56}{3584}\biggr| = \frac{1}{448} \leq \frac{1}{256} = \frac{1}{2}2^{-7}\]
que satisface la cota $|x - rd_n(x)| \leq \frac{1}{2}\beta^{-n}$. Por otra parte, el error en el segundo caso es 0, que evidentemente también satisface la cota.

\vspace{2mm}
\textbf{2. }

\vspace{2mm}
\textbf{(a) } La única raíz positiva de la ecuación es $l = \sqrt{a}$. La función de iteración del método de punto fijo es $g(x) = x + \lambda(x^2 - a)$. Vamos a elegir $\lambda \in \R$ de forma que sea $|g'(\sqrt{a})| < 1$. Tenemos que
$g'(x) = 1 + 2x\lambda$, luego
\[
\begin{aligned}[t]
|g'(\sqrt{a})| < 1 &\iff |1 + 2\sqrt{a}\lambda | <1 \iff -1 < 1+2\sqrt{a}\lambda < 1 \iff -1 < \sqrt{a}\lambda < 0 \\
&\iff -\frac{1}{\sqrt{a}} < \lambda < 0
\end{aligned}
\]
Por tanto, para cualquier $\lambda \in (-\frac{1}{\sqrt{a}},0)$, el método converge localmente hacia la raíz positiva de la ecuación. Para que el método sea de orden 2, vamos a tomar $\lambda \in \R$ de forma que $g'(\sqrt{a}) = 0$ y $g''(\sqrt{a}) \neq 0$.
\begin{itemize}
    \item $\displaystyle g'(\sqrt{a}) = 0 \iff 1 + 2\sqrt{a}\lambda = 0 \iff \lambda = -\frac{1}{2\sqrt{a}}$
    \item $\displaystyle g''(\sqrt{a}) \neq 0 \iff 2\lambda \neq 0 \iff \lambda \neq 0$
\end{itemize}
Por tanto, tomando $\lambda = -\frac{1}{2\sqrt{a}}$, el método es de orden 2. La constante de error asintótico es
\[C = \frac{|g''(\sqrt{a})|}{2} = \frac{|2\lambda|}{2} = \frac{1}{2\sqrt{a}}\]

\vspace{2mm}
\textbf{(b) } La función de iteración es $g(x) = \frac{1}{2} (x + \frac{a}{x})$. Entonces $g'(x) = \frac{1}{2}(1-\frac{a}{x^2})$, luego
\[|g'(\sqrt{a})| =  \biggl|\frac{1}{2}(1-\frac{a}{a}) \biggr| = 0 < 1\]
así que el método es localmente convergente y de orden al menos 2. Además,
\[g''(x) = \frac{a}{x^3} \neq 0 \ \forall \ x \neq 0\]
luego $g''(\sqrt{a}) \neq 0$ y el método es de orden 2. La constante de error asintótico es
\[C = \frac{|g''(\sqrt{a})|}{2} = \frac{1}{2\sqrt{a}}\]

\vspace{2mm}
\textbf{(c) } Sea $f(x) = x^2 - a$. Entonces $f'(x) = 2x, f''(x) = 2$. Si tomamos la semilla a la derecha de la raíz ($0 < \sqrt{a} < x_0$), entonces, en cualquier intervalo $J$ de la forma $[b,x_0]$ con $0 < b < \sqrt{a}$, se tiene que
\begin{itemize}
    \item $f(b)f(b) < 0 $ porque $0<b<a \implies b^2 -a^2< 0$ y $0<\sqrt{a}<x_0 \implies x_0^2-a^2>0$.
    \item $f'(x) \neq 0 \ \forall \ x \in J$, pues $0 \notin J$.
    \item $f''(x) \neq 0 \ \forall \ x \in J$.
    \item $f(x_0)f''(x_0) = 2f(x_0) \geq 0$.
\end{itemize}
Por tanto, la convergencia del método de Newton hacia la raíz positiva de la ecuación está asegurada.

\vspace{2mm}
\textbf{3. }

\vspace{2mm}
\textbf{(a) } Las diferencias divididas son
\begin{itemize}
    \item $f[x_0] = 1; f[x_1] = 2; f[x_2] = 3$
    \item $\displaystyle f[x_0,x_1] = \frac{1}{15}; f[x_1,x_2] = \frac{1}{65}$
    \item $\displaystyle f[x_0,x_1,x_2] = -\frac{1}{1560}$
\end{itemize}
luego \[p(x) = 1 + \frac{1}{15}(x-1) - \frac{1}{1560}(x-1)(x-16)\]
Como $2 = \sqrt[4]{16}, 3 = \sqrt[4]{81}$, el polinomio aproxima a la función $f(x) = \sqrt[4]{x}$ en el intervalo $[1,81]$.

\vspace{2mm}
\textbf{(b) } La aproximación de $\sqrt[4]{20}$ es $p(20) = 1 + \frac{19}{15} - \frac{76}{1560} = \frac{1560 + 1976-76}{1560} = \frac{173}{78}$, mientras que la aproximación de $\sqrt[4]{50}$ es $p(50) = 1+\frac{49}{15}-\frac{1666}{1560} = \frac{1560+5096-1666}{1560} = \frac{499}{156}$.

\vspace{2mm}
Esta última aproximación es menos exacta, ya que $\frac{499}{156}$ está cerca de 3, mientras que $\sqrt[4]{50}$ está cerca de $2'5$. Esto se debe a que $20$ se encuentra cerca de uno de los nodos de interpolación, mientras que $50$ está alrededor del centro del intervalo $[1,81]$, muy lejos de los nodos de interpolación.

\vspace{2mm}
\textbf{(c) } La cota del error vista en clase es
\[|\sqrt[4]{50}-p(50)| \leq \frac{M_3}{6}|50-1|^2|50-16|^2|50-81|^2 = \frac{21}{384} \cdot 49^2 \cdot 34^2 \cdot 31^2\]
donde
\[M_3 = \max_{x \in [1,81]} |f'''(x)| = \max_{x \in [1,81]} |\frac{21}{64}x^{-\frac{11}{4}}| = \frac{21}{64}\]
La cota obtenida es enorme como consecuencia de la gran distancia a la que se encuentran los nodos de interpolación.

\vspace{2mm}
\textbf{4. }

\vspace{2mm}
\textbf{(a) } Para que la fórmula sea de grado al menos 1, debe cumplirse
\[I_2(f) = \int_{-1}^1 1 \, dx \iff A + A = 2 \iff A = 1\]
Para que sea de grado al menos 2,
\[I_2(f) = \int_{-1}^1 x \, dx \iff x_0 + x_1 = 0 \iff x_0 = -x_1\]
Para que sea de grado al menos 3,
\[I_2(f) = \int_{-1}^1 x^2 \, dx \iff x_0^2 + x_1^2 = \frac{2}{3} \iff x_1^2 + x_1^2 = \frac{2}{3} \iff x_1 = \pm \frac{1}{\sqrt{3}}\]
En resumen, si $A = 1, x_0 = -\frac{1}{\sqrt{3}}, x_1 = \frac{1}{\sqrt{3}}$ ó si $A = 1, x_0 = \frac{1}{\sqrt{3}}, x_1 = -\frac{1}{\sqrt{3}}$, entonces la fórmula tiene grado máximo.

\vspace{2mm}
\textbf{(b) } Veamos si la fórmula es de grado de exactitud 3. Si $A = 1, x_0 = -\frac{1}{\sqrt{3}}, x_1 = \frac{1}{\sqrt{3}}$, entonces
\begin{itemize}
    \item $\displaystyle \int_{-1}^1 x^3 \, dx = 0$
    \item $\displaystyle I_2(f) = -\frac{1}{3\sqrt{3}} + \frac{1}{3\sqrt{3}} = 0$
\end{itemize}
Lo mismo se obtiene en el caso $A = 1, x_0 = \frac{1}{\sqrt{3}}, x_1 = -\frac{1}{\sqrt{3}}$. Por tanto, la fórmula es de grado al menos 4. Veamos si es de grado 4: si $A = 1, x_0 = -\frac{1}{\sqrt{3}}, x_1 = \frac{1}{\sqrt{3}}$, entonces
\begin{itemize}
    \item $\displaystyle \int_{-1}^1 x^4 \, dx = \frac{2}{5}$
    \item $\displaystyle I_2(f) = \frac{1}{9} + \frac{1}{9} \neq \frac{2}{5}$
\end{itemize}
Lo mismo se obtiene para $A = 1, x_0 = \frac{1}{\sqrt{3}}, x_1 = -\frac{1}{\sqrt{3}}$. Por tanto, la fórmula es de grado 4. Se trata de la fórmula de Gauss con 2 puntos.

\vspace{2mm}
\textbf{(c) } Sea $f \colon [-1,1] \to \R$, $f(x) = x^4$. Entonces $f \in \mathcal{C}^4([-1,1])$, así que existe $\xi \in [-1,1]$ con
\[\int_{-1}^1 x^4 \, dx = I_2(f) + Kf^{(4)}(\xi) \iff \frac{2}{5} = \frac{2}{9} + 24K \iff K = \frac{\frac{2}{5}-\frac{2}{9}}{24} = \frac{1}{135}\]
\end{document}