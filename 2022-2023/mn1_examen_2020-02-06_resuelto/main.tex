% Actualizado a día 2023-02-23

\documentclass[12pt]{report}

\usepackage{graphicx}
\usepackage[a4paper, total={6.5in, 9in}, top=1.5in]{geometry}
\usepackage[utf8]{inputenc}
\usepackage[spanish]{babel}
\decimalpoint
\usepackage{amsmath,amsfonts,amssymb,amsthm}
\usepackage{fancyhdr}

\parindent=0pt

% Shortcuts:
\newcommand{\R}{\mathbb R}
\newcommand{\N}{\mathbb N}
\newcommand{\Z}{\mathbb Z}
\newcommand{\Q}{\mathbb Q}

\pagestyle{fancy}
\fancyhead[L]{\textbf{Métodos Numéricos I}}
\fancyhead[R]{\textbf{Examen final - 2020/02/06}}

\begin{document}
\textbf{1. } Sean $x = 2^n, y = 2^{-n}$ números reales almacenables en el sistema de representación en punto flotante del ordenador. Como $fl(x) = rd_{52}(1) \cdot 2^n, fl(y) = rd_{52}(1) \cdot 2^{-n}$, entonces $fl(x) = x, fl(y) = y$. Por otra parte,
\[x + y = 2^n + 2^{-n} = (\overbrace{10 \mathellipsis 0}^{n+1}{'}  \overbrace{0\mathellipsis01}^{n})_\beta = (1'\overbrace{0 \mathellipsis 0 1}^{2n})_\beta \cdot 2^n\]
Distinguimos dos casos:
\begin{itemize}
    \item Si $2n \geq 54$, es decir, $n \geq 27$, entonces $rd_{52}((1'\overbrace{0 \mathellipsis 0 1}^{2n})_\beta) = 1$, luego
    \[x \oplus y = fl(fl(x) + fl(y)) = fl(x + y) = (1)_\beta \cdot 2^n = 2^n\]
    \item Si $2n \leq 52$, es decir, $n \leq 26$, entonces $rd_{52}((1'\overbrace{0 \mathellipsis 0 1}^{2n})_\beta) = (1'\overbrace{0 \mathellipsis 0 1}^{2n})_\beta$, luego
    \[x \oplus y = fl(fl(x) + fl(y)) = fl(x + y) = (1'\overbrace{0 \mathellipsis 0 1}^{2n})_\beta \cdot 2^n = 2^n + 2^{-n}\]
\end{itemize}
En cuanto al error relativo,
\begin{itemize}
    \item Si $n \geq 27$, 
    \[\Biggl|\frac{x+y - (x \oplus y)}{x+y} \Biggr| = \Biggl|\frac{2^n + 2^{-n} - 2^n}{2^n + 2^{-n}} \Biggr| = \frac{2^{-n}}{2^n + 2^{-n}} = \frac{1}{2^{2n}+1}\]
    \item Si $n \leq 26$, 
    \[\Biggl|\frac{x+y - (x \oplus y)}{x+y} \Biggr| = \Biggl|\frac{2^n + 2^{-n} - 2^n - 2^{-n}}{2^n + 2^{-n}} \Biggr| = 0\]
\end{itemize}

\vspace{2mm}
\textbf{2. }

\vspace{2mm}
\textbf{(a)} Tenemos que
\[g'(x) = 1 - \frac{f'(x)}{f'(l)}\]
Esto significa que
\begin{itemize}
    \item $|g'(l)| = |1 - 1| = 0 < 1$, así que el método es localmente convergente.
    \item $g'(l) = 0$, así que el métodoo es de al menos orden 2.
\end{itemize}
Además,
\[g''(x) = - \frac{f''(x)}{f'(l)}\]
Por tanto, si fuese
\[g''(l) = 0 \iff -\frac{f''(l)}{f'(l)} = 0 \iff f''(l) = 0\]
entonces el método sería de al menos orden 3. En cambio, si $f''(l) \neq 0$, entonces $g''(l) \neq 0$ y el método sería de orden 2.

\vspace{2mm}
Si $p \geq 3$ es el orden del método, la constante de error asintótico sería
\[C = \frac{|g^{(p)}(l)|}{p!}\]
Y si fuese $p = 2$, entonces
\[C = \frac{|g''(l)|}{2} = \Biggl|\frac{f''(l)}{2f'(l)} \Biggr|\]

\vspace{2mm}
\textbf{(b)} La única raíz de $f$ en $J$ es $l = \pi$. Como 
\begin{itemize}
    \item $\displaystyle g''(\pi) = -\frac{f''(\pi)}{f'(\pi)} = \frac{\sin{\pi}}{\cos{\pi}} = 0$
    \item $\displaystyle g'''(\pi) = -\frac{f'''(\pi)}{f'(\pi)} = \frac{\cos{\pi}}{\cos{\pi}} = 1 \neq 0$
\end{itemize}
entonces el método es de orden $3$. La constante de error asintótico es
\[C = \frac{|g'''(\pi)|}{3!} = \frac{1}{6}\]

\vspace{2mm}
\textbf{(c)} Como $g$ es de clase $\mathcal{C}^1$ en cualquier intervalo cerrado y acotado y además
\[|g'(x)| = |1 - \frac{f'(x)}{f'(\pi)}| = |1 + \cos{x}| < 1 \iff \cos{x} < 0 \iff x \in \biggl(\frac{\pi}{2}k, \frac{3\pi}{2}k \biggr), k \in \Z\]
entonces no podemos asegurar la contractividad de $g$ en $J$ (pues $\cos(\frac{\pi}{2}) = \cos(\frac{3\pi}{2}) = 0$), pero sí en $J' \subset (\frac{\pi}{2},\frac{3\pi}{2})$.

\vspace{2mm}
\textbf{(d) } La constante de contractividad en $J'$ es $C = 1 - \frac{\sqrt{2}}{2}$, ya que
\[|g'(x) = |1+\cos{x}| \leq |1 + \cos{\frac{3\pi}{4}}| = |1 + \cos{\frac{5\pi}{4}}| = |1 - \frac{\sqrt{2}}{2}| = 1 - \frac{\sqrt{2}}{2} < 1\]
Usaremos la cota
\[|x_n - l| \leq C^n(b-a) = \biggl(1-\frac{\sqrt{2}}{2} \biggr)^n \cdot \frac{\pi}{2}\]
Por tanto,
\[\biggl(1-\frac{\sqrt{2}}{2} \biggr)^n \cdot \frac{\pi}{2} \leq \frac{1}{2}10^{-10} \iff \biggl(1-\frac{\sqrt{2}}{2} \biggr)^n \leq \frac{10^{-10}}{\pi} \iff n \geq \frac{\log(\frac{10^{-10}}{\pi})}{\log({1-\frac{\sqrt{2}}{2}})} \approx 19'68\]
En 20 iteraciones, podemos asegurar 10 cifras decimales exactas.

\vspace{2mm}
\textbf{3. } 

\vspace{2mm}
\textbf{(a) } Las diferencias divididas son
\begin{itemize}
    \item $f[x_0] = 0; f[x_1] = 1; f[x_2] = 0; f[x_3] = -1; f[x_4] = 0$
    \item $\displaystyle f[x_0,x_1] = \frac{2}{\pi}; f[x_1,x_2] = -\frac{2}{\pi}; f[x_2,x_3] = -\frac{2}{\pi}; f[x_3,x_4] = \frac{2}{\pi}$
    \item $\displaystyle f[x_0,x_1,x_2] = -\frac{4}{\pi^2}; f[x_1,x_2,x_3] = 0; f[x_2,x_3,x_4] = \frac{4}{\pi^2}$
    \item $\displaystyle f[x_0,x_1,x_2,x_3] = \frac{8}{3\pi^3}; f[x_1,x_2,x_3,x_4] = \frac{8}{3\pi^3}$
    \item $\displaystyle f[x_0,x_1,x_2,x_3,x_4] = 0$
\end{itemize}
Por tanto,
\[P_4(x) = \frac{2}{\pi}x - \frac{4}{\pi^2}x(x-\frac{\pi}{2}) + \frac{8}{3\pi^3}x(x-\frac{\pi}{2})(x-\pi)\]

\vspace{2mm}
\textbf{(b) } El polinomio que interpola los datos $(0,f(0)),(\frac{\pi}{2},f(\frac{\pi}{2})),(\pi,f(\pi))$
es
\[P_2^1(x) = \frac{2}{\pi}x -\frac{4}{\pi^2}x(x-\frac{\pi}{2})\]
y el que interpola los datos $(\pi,f(\pi)),(\frac{3\pi}{2},f(\frac{3\pi}{2})),(2\pi,f(2\pi))$ es
\[P_2^2(x) = -\frac{2}{\pi}(x-\pi) + \frac{4}{\pi^2}(x-\pi)(x-\frac{3\pi}{2})\]
Por tanto,
\[P_2(x) = 
\begin{cases}
\frac{2}{\pi}x -\frac{4}{\pi^2}x(x-\frac{\pi}{2}) & $si$ \ x \in [0,\pi] \\
-\frac{2}{\pi}(x-\pi) + \frac{4}{\pi^2}(x-\pi)(x-\frac{3\pi}{2}) & $si$ \ x \in [\pi,2\pi]
\end{cases}
\]

\vspace{2mm}
\textbf{(c) } El error que se comete al aproximar $f$ por $P_4$ es
\[|f(x) - P_4(x)| \leq \frac{M_5}{5!}(b-a)^5\]
donde $\displaystyle b-a=2\pi, M_5 = \max_{x \in [0,2\pi]}|f^{(5)}(x)| = 1$. Por tanto,
\[|f(x) - P_4(x)| \leq \frac{32\pi^5}{120} = \frac{4\pi^5}{15}\]
Por otra parte, si $x \in [0,\pi]$, el polinomio $P_2^1$ interpola los extremos del intervalo y su punto medio, así que la cota óptima sería
\[|f(x) - P_2(x)| = |f(x) - P_2^1(x)| \leq \frac{M_3}{72\sqrt{3}}(b-a)^3 = \frac{\pi^3}{72\sqrt{3}}\]
Si $x \in [\pi,2\pi]$, la cota obtenida sería exactamente la misma, así que la cota anterior es uniforme en el intervalo $[0,2\pi]$.

\vspace{2mm}
\textbf{(d) } En las condiciones del enunciado, consideramos el polinomio $p$ que interpola los datos
\[(x_0,f(x_0)),(x_1,f(x_1)), \mathellipsis, (x_n,f(x_n))\]
Sea $P \colon [a,b] \to \R$ definida por $P(x) = f(x) - p(x)$, que es de clase $\mathcal{C}^n([a,b])$ y verifica $P(x_i) = 0 \ \forall \ i = 0,\mathellipsis,n$. 

\vspace{2mm}
Por el teorema de Rolle, en cada subintervalo $[x_i, x_{i+1}]$ (con $i = 0, \mathellipsis, n-1$), existe $\xi_i^0 \in (x_i,x_{i+1})$ tal que $P'(\xi_i^0) = 0$. Aplicando el teorema de Rolle a $P'$, en cada intervalo $[\xi_i^0,\xi_{i+1}^0]$ ($i = 0, \mathellipsis, n-2$), existe $\xi_i^1 \in (\xi_i^0,\xi_{i+1}^0) \subset (\min{(x_0,\mathellipsis,x_n)},\max{(x_0,\mathellipsis,x_n)})$ tal que $P''(\xi_i^1) = 0$. Repitiendo este razonamiento $n$ veces, podemos concluir que existe $\xi \in (\min{(x_0,\mathellipsis,x_n)},\max{(x_0,\mathellipsis,x_n)})$ tal que $P^{(n)}(\xi) = 0$. 

\vspace{2mm}
Por otra parte, como $p$ es un polinomio de grado $n$ y el coeficiente de $x^n$ en $p$ es $f[x_0,x_1,\mathellipsis,x_n]$, entonces $p^{(n)}(x) = f[x_0,x_1,\mathellipsis,x_n]n!$, luego
\[P^{(n)}(\xi) = 0 \iff f^{(n)}(\xi) = p^{(n)}(\xi) \iff f[x_0,x_1,\mathellipsis,x_n] = \frac{f^{(n)}(\xi)}{n!}\]

\vspace{2mm}
\textbf{4. }

\vspace{2mm}
\textbf{(a) } Veamos si es exacta para $f(x) = 1$.
\begin{itemize}
    \item $\displaystyle \int_{-1}^1 1 \, dx = 2$
    \item $\displaystyle I_3(f) = \frac{5}{9} + \frac{8}{9} + \frac{5}{9} = 2$
\end{itemize}
Veamos si es exacta para $f(x) = x$.
\begin{itemize}
    \item $\displaystyle \int_{-1}^1 x \, dx = 0$
    \item $\displaystyle I_3(f) = \frac{5}{9} \biggl(-\sqrt{\frac{3}{5}} \biggr) + \frac{8}{9} \cdot 0 + \frac{5}{9} \sqrt{\frac{3}{5}} = 0$
\end{itemize}
Veamos si es exacta para $f(x) = x^2$.
\begin{itemize}
    \item $\displaystyle \int_{-1}^1 x^2 \, dx = \frac{2}{3}$
    \item $\displaystyle I_3(f) = \frac{5}{9} \cdot \frac{3}{5} + \frac{8}{9} \cdot 0 + \frac{5}{9} \cdot \frac{3}{5} = \frac{2}{3}$
\end{itemize}
Veamos si es exacta para $f(x) = x^3$.
\begin{itemize}
    \item $\displaystyle \int_{-1}^1 x^3 \, dx = 0$
    \item $\displaystyle I_3(f) = \frac{5}{9} \biggl(-\sqrt{\frac{27}{125}} \biggr) + \frac{8}{9} \cdot 0 + \frac{5}{9} \sqrt{\frac{27}{125}} = 0$
\end{itemize}
Veamos si es exacta para $f(x) = x^4$.
\begin{itemize}
    \item $\displaystyle \int_{-1}^1 x^4 \, dx = \frac{2}{5}$
    \item $\displaystyle I_3(f) = \frac{5}{9} \cdot \frac{9}{25} + \frac{8}{9} \cdot 0 + \frac{5}{9} \cdot \frac{9}{25} = \frac{2}{5}$
\end{itemize}
Veamos si es exacta para $f(x) = x^5$.
\begin{itemize}
    \item $\displaystyle \int_{-1}^1 x^5 \, dx = 0$
    \item $\displaystyle I_3(f) = \frac{5}{9} \biggl(-\sqrt{\frac{243}{3125}} \biggr) + \frac{8}{9} \cdot 0 + \frac{5}{9} \sqrt{\frac{243}{3125}} = 0$
\end{itemize}
Veamos si es exacta para $f(x) = x^6$.
\begin{itemize}
    \item $\displaystyle \int_{-1}^1 x^6 \, dx = \frac{2}{7}$
    \item $\displaystyle I_3(f) = \frac{5}{9} \cdot \frac{27}{125} + \frac{8}{9} \cdot 0 + \frac{5}{9} \cdot \frac{27}{125} = \frac{6}{25} \neq \frac{2}{7}$
\end{itemize}
Por tanto, la fórmula es de grado 5. Se trata de la fórmula de Gauss con 3 puntos.

\vspace{2mm}
\textbf{(b) } Sea $f(x) = x^6$ (que es de clase 6 en $[-1,1]$). Entonces existe $\xi \in (-1,1)$ tal que 
\[\int_{-1}^1 x^6 \, dx = I_3(f) + Kf^{(6)}(\xi) \iff \frac{2}{7} = \frac{6}{25} + K \cdot 720 \iff K = \frac{\frac{2}{7} - \frac{6}{25}}{720} = \frac{8}{720 \cdot 175} = \frac{1}{15750}\]

\vspace{2mm}
\textbf{(c) } Sea $f(x) = x^7 + x^6$. Entonces
\[
I_3(f) = \frac{5}{9}\biggl(-\sqrt{\frac{2187}{78125}} + \frac{27}{125} \biggr) + \frac{5}{9}\biggl(\sqrt{\frac{2187}{78125}} + \frac{27}{125} \biggr) = \frac{6}{25}
\]
La integral exacta es
\[\int_{-1}^1 (x^7 + x^6) \, dx = \biggl[\frac{x^8}{8} + \frac{x^7}{7} \biggr]_{-1}^1 = \frac{1}{8} + \frac{1}{7} - \frac{1}{8} + \frac{1}{7} = \frac{2}{7}\]
El error cometido es 
\[\biggl| \frac{2}{7} - \frac{6}{25} \biggr| = \frac{8}{175}\]
y la cota que proporciona el apartado anterior,
\[\biggl|\int_{-1}^1 f(x) \, dx - I_3(f) \biggr| \leq \frac{M_6}{15750} = \frac{8}{25}\]
donde 
\[M_6 = \max_{x \in [-1,1]} |f^{(6)}(x)| = \max_{x \in [-1,1]} |5040x| = 5040\]
Evidentemente, el error exacto satisface esta cota.

\vspace{2mm}
\textbf{(d) } Los nuevos pesos son
\[\tilde{\alpha}_0 = \frac{b-a}{2}\alpha_0, \ \tilde{\alpha}_1 = \frac{b-a}{2}\alpha_1, \ \tilde{\alpha}_2 = \frac{b-a}{2}\alpha_2\]
y los nuevos nodos,
\[\tilde{x}_0 = a + \frac{b-a}{2}(x_0 + 1), \ \tilde{x}_1 = a + \frac{b-a}{2}(x_1 + 1), \ \tilde{x}_2 = a + \frac{b-a}{2}(x_2 + 1)\]
Por tanto, 
\[
\begin{aligned}[t]
I_3(f) &= \frac{5(b-a)}{18}f \biggl(a + \frac{b-a}{2} \biggl(1-\sqrt{\frac{3}{5}} \biggr) \biggr) + \frac{4(b-a)}{9}f\biggl( a + \frac{b-a}{2} \biggr)  \\
&+ \frac{5(b-a)}{18}f \biggl(a + \frac{b-a}{2} \biggl(1+\sqrt{\frac{3}{5}} \biggr) \biggr)
\end{aligned}
\]
\end{document}