\documentclass[12pt]{report}

\usepackage{graphicx}
\usepackage[a4paper, total={6.5in, 9in}]{geometry}
\usepackage[utf8]{inputenc}
\usepackage[spanish]{babel}
\decimalpoint
\usepackage{amsmath,amsfonts,amssymb,amsthm}
\usepackage{fancyhdr}
\usepackage{multicol}
\usepackage{fbox}
\usepackage{mathtools}

\parindent=0pt

% Shortcuts:
\newcommand{\R}{\mathbb R}
\newcommand{\N}{\mathbb N}
\newcommand{\Z}{\mathbb Z}
\newcommand{\Q}{\mathbb Q}

\begin{document}

\small

\textbf{Geometría Diferencial de Curvas y Superficies} \hfill \textbf{Examen de febrero de 2022}
\linebreak

\textbf{1. } \textit{Sea $\alpha \colon I \to \R^3$ una curva parametrizada por el arco de curvatura nunca nula. Demostrar que todos los planos normales de $\alpha$ pasan por un punto fijo si y solo si la traza de $\alpha$ está contenida en una esfera.}

\begin{itemize}
    \item[{\fbox[rb]{$\Rightarrow$}}] Supóngase que todos los planos normales de $\alpha$ pasan por un punto fijo $p_0 = (x_0,y_0,z_0)$. La ecuación del plano normal de $\alpha$ en el punto $\alpha(s)$ es
    \[A(x-x_0)+B(y-y_0)+C(z-z_0) = 0\]
    siendo $\alpha'(s) = T(s) = (A,B,C)$. Como el punto $\alpha(s) = (x(s),y(s),z(s))$ pertenece al plano, entonces
    \[A(x(s)-x_0)+B(y(s)-y_0)+C(z(s)-z_0) = 0 \iff \langle \alpha'(s),\alpha(s)-p_0 \rangle= 0\]
    Esto significa que la función $s \mapsto \langle \alpha(s)-p_0,\alpha(s)-p_0 \rangle = ||\alpha(s)-p_0||^2$ es una constante no negativa. Llamando $r$ a esta constante, se tiene que $||\alpha(s)-p_0||^2 = r$, de donde se deduce que la traza de $\alpha$ está contenida en la esfera de centro $p_0$ y radio $\sqrt{r}$.
    \item[{\fbox[rb]{$\Leftarrow$}}] Supóngase que la traza de $\alpha$ está contenida en una esfera de centro $c$ y radio $r$, es decir, que $||\alpha(s)-c||^2 = r^2$ para todo $s \in I$, siendo $c \in \R^3$, $r \in (0,\infty)$ fijos. Derivando,
    \[2\langle \alpha'(s),\alpha(s)-c \rangle = 0 \iff \langle T(s),\alpha(s)-c \rangle = 0\]
    Llamando $T(s) = (A,B,C), \alpha(s) = (x(s),y(s),z(s))$ $c = (x_0,y_0,z_0)$, se tiene que
    \[A(x(s)-x_0)+B(y(s)-y_0)+C(z(s)-z_0) = 0\]
    de donde se deduce que $c$ está en el plano de ecuación
    \[A(x-x(s))+B(y-y(s))+C(z-z(s)) = 0\]
    que no es más que el plano normal de $\alpha$ en el punto $\alpha(s)$.
\end{itemize}

\textbf{2. } \textit{Demuéstrese que una curva parametrizada diferenciable regular admite una reparametrización por la longitud de arco.}

\vspace{2mm}
Sea $\alpha \colon (a,b) \to \R^3$ una curva parametrizada diferenciable regular. Sea $u_0 \in (a,b)$ y sea
\[
\begin{aligned}[t]
    s \colon (a,b) &\longrightarrow (c,d)  \\
    u &\longmapsto s(u) = \int_{u_0}^u ||\alpha'(t)|| \, dt
\end{aligned}
\]
Se tiene que $s$ es diferenciable y $s'(u) = ||\alpha'(u)|| > 0$ para todo $u \in (a,b)$. Por el teorema de la función inversa, $s$ es un difeomorfismo cuya inversa es $t \colon (c,d) \to (a,b)$, que además verifica
\[t'(u) = \frac{1}{s'(t(u))} \iff t'(u)s'(t(u)) = 1\]
para todo $u \in (c,d)$. Nótese que además $t'(u) > 0$ por ser $s'(t(u)) > 0$ para todo $u \in (c,d)$. Se considera la reparametrización $\beta \colon (c,d) \longrightarrow \R^3$ definida por $\beta(u) = \alpha(t(u))$.
Entonces
\[||\beta'(u)|| = ||\alpha'(t(u))t'(u)|| = |t'(u)| \, ||\alpha'(t(u))|| = t'(u)s'(t(u)) = 1\]
así que $\alpha$ admite una reparametrización por la longitud de arco.
\pagebreak

\vspace{4mm}
\textbf{3. } \textit{Sea $S$ una superficie regular y sea $p \in S$. Dados $v, w \in T_pS$ no nulos, se dirá que $v$ y $w$ definen} direcciones conjugadas \textit{si $\mathbb{I}_p(v,w) = 0$.}
\begin{itemize}
    \item[\textit{(a)}] \textit{Demuéstrese que si $v$ y $w$ definen direcciones conjugadas, ocurre lo mismo entre cualquier par de múltiplos no nulos de $v$ y $w$.}
    \item[\textit{(b)}] \textit{Demuéstrese que si $p$ es punto llano, entonces todo par de direcciones son conjugadas.}
    \item[\textit{(c)}] \textit{Demuéstrese que si $p$ es punto umbílico no llano, entonces todo par de direcciones ortogonales son conjugadas. }
    \item[\textit{(d)}] \textit{Demuéstrese que si $p$ no es punto umbílico, entonces las direcciones principales en $p$ son siempre conjugadas.}
    \item[\textit{(e)}] \textit{Demuéstrese que una dirección asintótica es conjugada de sí misma.}
\end{itemize}

\textbf{(a) } Sean $v,w \in T_pS$ tales que $\mathbb{I}_p(v,w) = 0$ y sean $\lambda v, \mu w$ múltiplos suyos no nulos. Entonces
\[\mathbb{I}_p(\lambda v, \mu w) = \langle \mathcal{S}_p(\lambda v), \mu w \rangle = \lambda \mu \, \langle \mathcal{S}_p(v), w \rangle = \lambda \mu \, \mathbb{I}_p(v,w) = 0\]
luego $\lambda v$ y $\mu w$ también definen direcciones conjugadas.

\vspace{2mm}
\textbf{(b) } Sea $p \in S$ un punto llano, es decir, con $k = 0$ y $\mathcal{S}_p = 0$. Sean $v,w \in T_pS$. Entonces
\[\mathbb{I}_p(v,w) = \langle \mathcal{S}_p(v),w \rangle = \langle 0, w\rangle =0\]
luego $v,w$ definen direcciones conjugadas.

\vspace{2mm}
\textbf{(c) }
Sea $p \in S$ un punto umbílico no llano y sean $v,w \in T_pS$ tales que $\langle v,w \rangle$. Como $p$ es punto umbílico, la segunda forma fundamental es proporcional a la métrica, luego
\[\mathbb{I}_p(v,w) = c \langle v,w \rangle = 0\]
para alguna constante $c \in \R$, así que $v,w$ son direcciones conjugadas.

\vspace{2mm}
\textbf{(d) } Si $p$ no es punto umbílico, las direcciones principales $e_1,e_2$ son distintas y ortogonales, luego
\[\mathbb{I}_p(e_1,e_2) = \langle \mathcal{S}_p(e_1),e_2 \rangle = k_1(p)\langle e_1,e_2 \rangle = 0\]
luego las direcciones principales definen direcciones conjugadas.

\vspace{2mm}
\textbf{(e) } Sea $w \in T_pS$ una dirección asintótica. Por definición de dirección asintótica, se tiene que $\mathbb{I}_p(w,w) = 0$, y por tanto la dirección es conjugada de sí misma.

\vspace{4mm}
\textbf{4. } \textit{Considérense las funciones $f,g \colon U \to \R$ definidas por
\[f(x,y,z) = x(x-1)+y^2+z^2 \qquad \qquad g(x,y,z) = x^2+y(y-1)+z^2
\]}
\begin{itemize}
    \item[\textit{(a)}] \textit{Pruébese que $S_1 = f^{-1}\{0\}$ y $S_2 = g^{-1}\{0\}$ son superficies regulares.}
    \item[\textit{(b)}] \textit{Pruébese que en cualquier punto $p \in S_1 \cap S_2$ las superficies se cortan ortogonalmente, es decir, sus campos normales unitarios son ortogonales.}
\end{itemize}

\textbf{(a) } Hay que demostrar que $0$ es valor regular de $f$ y $g$. Las matrices jacobianas en cada punto son
\[Jf_p =\begin{pmatrix}
    2x-1 & 2y & 2z
\end{pmatrix}_p \qquad \qquad Jg_p =\begin{pmatrix}
    2x & 2y-1 & 2z
\end{pmatrix}_p \]
La primera solo se anula en $(\frac{1}{2},0,0)$, pero $f(\frac{1}{2},0,0) \neq 0$, así que $(\frac{1}{2},0,0) \notin f^{-1}\{0\}$ y por tanto $0$ es valor regular de $f$. La segunda solo se anula en $(0,\frac{1}{2},0)$, pero $g(0,\frac{1}{2},0) \neq 0$, así que $(0,\frac{1}{2},0) \notin g^{-1}\{0\}$ y por tanto $0$ es valor regular de $g$. Esto prueba que tanto $S_1$ como $S_2$ son superficies regulares.

\vspace{2mm}
\textbf{(b) }
Un campo normal unitario definido en toda la superficie $S_1$ es
\[\mathcal{N}_p^1 = \frac{(f_x,f_y,f_z)_p}{||(f_x,f_y,f_z)_p||} = \frac{1}{\sqrt{(2x-1)+4y^2+4z^2}}(2x-1,2y,2z)\]
Un campo normal unitario definido en toda la superficie $S_2$ es
\[\mathcal{N}_p^2 = \frac{(g_x,g_y,g_z)_p}{||(g_x,g_y,g_z)_p||} = \frac{1}{\sqrt{4x^2+(2y-1)^2+4z^2}}(2x,2y-1,2z)\]
A partir de ahora se van a ignorar las dos raíces cuadradas de arriba con el fin de ahorrar escritura. Esto no supone ningún inconveniente porque va a resultar que los vectores son ortogonales. Sea $p =(x,y,z)\in S_1 \cap S_2$. Entonces
\[
f(p)=0 \iff y^2 = -z^2-x(x-1)
\]
\[
g(p)=0 \iff x^2 = -z^2-y(y-1)
\]
luego
\[
\begin{aligned}[t]
\langle \mathcal{N}_p^1,\mathcal{N}_p^2 \rangle &= 4x^2-2x+4y^2-2y+4z^2 \\
&=-4z^2-4y(y-1)-2x-4z^2-4x(x-1)-2y+4z^2 \\
&=-4z^2-4y(y-1)-2x-4z^2-4x^2+4x-2y+4z^2 \\
&=-4(x^2+y(y-1)+z^2)-2x+4x-2y \\
&= 2(x-y)
\end{aligned}
\]
Ahora bien, como se tiene que
\[y^2 = -z^2-x^2+x \iff x = x^2+y^2+z^2\]
\[x^2 = -z^2-y^2+y \iff y = x^2+y^2+z^2\]
entonces $\langle \mathcal{N}_p^1,\mathcal{N}_p^2 \rangle = 2(x-y) = 0$ y por tanto las superficies se cortan ortogonalmente.
\end{document}