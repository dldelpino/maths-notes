\documentclass[10pt, letterpaper]{article}

\usepackage{preamble}

\begin{document}

\begin{multicols}{2}
        \textbf{\fontsize{24 pt}{24 pt}\selectfont David López del Pino}

        \vspace{0.4 cm}

        \normalsize

        \hspace{0.4 mm}\mbox{{\color{black}\footnotesize\faMapMarker*}\hspace*{0.16 cm}Málaga, España}%

        \mbox{\hrefWithoutArrow{mailto:lopezdelpinodavid@gmail.com}{\color{black}{\footnotesize\faEnvelope[regular]}\hspace*{0.13cm}lopezdelpinodavid@gmail.com}}%

        \mbox{\hrefWithoutArrow{tel:+34-622-23-23-87}{\color{black}{\footnotesize\faPhone*}\hspace*{0.13cm}622 23 23 87}}%

        % \mbox{\hrefWithoutArrow{https://yourwebsite.com/}{\color{black}{\footnotesize\faLink}\hspace*{0.13cm}yourwebsite.com}}%
        % \kern 0.25 cm%
        % \AND%
        % \kern 0.25 cm%
        \mbox{\hrefWithoutArrow{https://linkedin.com/in/david-lopez-del-pino}{\color{black}{\footnotesize\faLinkedinIn}\hspace*{0.13cm}david-lopez-del-pino}}%

        \mbox{\hrefWithoutArrow{https://github.com/dldelpino}{\color{black}{\footnotesize\faGithub}\hspace*{0.13cm}dldelpino}}%

        \vfill\null
        \columnbreak

        \begin{flushright}
            \begin{tikzpicture}
                \node [inner sep=0pt,,outer sep=0pt,clip,rounded corners=0.5cm] (pict) at (0,0) {\includegraphics[width = 3.5cm]{img.png}};
                \node[draw=gray,line width = 1.2pt,fit=(pict),rounded corners=.55cm,inner sep=0pt]    {};
            \end{tikzpicture}
        \end{flushright}
\end{multicols}

    \section{Sobre mí}
        \begin{onecolentry}
            Recién graduado en Matemáticas con sólidos conocimientos en desarrollo en Python, análisis de datos y resolución de problemas. Busco incorporarme a un equipo donde pueda ganar experiencia y desarrollar mi carrera profesional.
            
            \vspace{0.2 cm}

            Me considero una persona proactiva, responsable y organizada, con gran capacidad de aprendizaje, comunicación y trabajo en equipo. Estoy dispuesto a mejorar mis habilidades y adquirir conocimientos para afrontar nuevos retos.

        \end{onecolentry}

    \section{Educación}

        \begin{twocolentry}{
            
        \textit{Septiembre de 2021 – Junio de 2025}}
            \textbf{Universidad de Málaga}

            \textit{Grado en Matemáticas}
        \end{twocolentry}

        \vspace{0.10 cm}
        \begin{onecolentry}
            \begin{highlights}
                \item \emph{Nota media:} 8,94/10 % (\href{https://example.com}{a link to somewhere})
                \item \emph{Trabajo de Fin de Grado:} Resultados sobre convergencia de series de Fourier
                \item Matrícula de honor en 11 asignaturas
            \end{highlights}
        \end{onecolentry}
    
    \section{Habilidades técnicas}
 
        \begin{twocolentry}{
            
        \textit{2022 – 2025}}
            \textbf{Programación en Python}

            \textit{Universidad de Málaga}
        \end{twocolentry}

        \vspace{0.10 cm}
        \begin{onecolentry}
            \begin{highlights}
                \item Resolución numérica de ecuaciones en derivadas parciales mediante el método de diferencias finitas y el método de elementos finitos.
                \item Visualización de soluciones de ecuaciones diferenciales y análisis de la precisión con NumPy, SciPy y Matplotlib.
                \item Extracción de datos de páginas web utilizando la BeautifulSoup.
                \item Análisis y manipulación de datos y estructuras tabulares con Pandas.
            \end{highlights}
        \end{onecolentry}

    \section{Tecnologías}

        \begin{onecolentry}
            \textbf{Lenguajes de programación:} Python, Scala, Haskell, C++, C, Java, JavaScript, SQL
        \end{onecolentry}

        \vspace{0.2 cm}

        \begin{onecolentry}
            \textbf{Otras tecnologías:} HTML, CSS, React, TypeScript, LaTeX, Wolfram Mathematica, Git, MySQL
        \end{onecolentry}

    \section{Idiomas}

        \begin{twocolentry}{
            
        \textit{Septiembre de 2023}}
            \textbf{Inglés (C2)}

            \textit{Cambridge English: Advanced}
        \end{twocolentry}

        \vspace{0.10 cm}
        \begin{onecolentry}
            \begin{highlights}
                \item \emph{Nota:} 204/210 % (\href{https://example.com}{a link to somewhere})
            \end{highlights}
        \end{onecolentry}

        \vspace{0.2 cm}
    
        \begin{twocolentry}{
            
        \textit{Abril de 2021}}
            \textbf{Francés (B2)}

            \textit{Diplôme d'Études en Langue Française (DELF)}
        \end{twocolentry}

        \vspace{0.10 cm}
        \begin{onecolentry}
            \begin{highlights}
                \item \emph{Nota:} 87.5/100 % (\href{https://example.com}{a link to somewhere})
            \end{highlights}
        \end{onecolentry}

        \vspace{0.2 cm}

        \begin{onecolentry}
            \textbf{Español (lengua materna)}
        \end{onecolentry}

\end{document}