\documentclass[11pt]{report}

\usepackage{graphicx}
\usepackage[a4paper, right = 0.9in, left = 0.9in, top = 1in, bottom = 1in]{geometry}
\usepackage[utf8]{inputenc}
\usepackage[spanish]{babel}
\decimalpoint
\usepackage{amsmath,amsfonts,amssymb,amsthm}
\usepackage{fancyhdr}
\usepackage{multicol}
\usepackage{fbox}
\usepackage[partialup]{kpfonts}

% Shortcuts:
\newcommand{\R}{\mathbb R}
\newcommand{\N}{\mathbb N}
\newcommand{\Z}{\mathbb Z}
\newcommand{\Q}{\mathbb Q}

\begin{document}

\begin{center}
    \textbf{Examen final de Ecuaciones Diferenciales II} \\
    \textbf{Viernes, 23 de junio de 2023}
\end{center}

\hrule

\vspace{4mm}

\noindent 1. \begin{itemize}
    \item[\textit{(a)}] \textit{Supongamos que $A \colon \R \to \mathcal{M}_n(\R)$ es continua y $\omega$-periódica, con $\omega \in (0,\infty)$. Consideramos el sistema}
    \[(H) \quad x'=A(t)x\]
    \begin{itemize}
        \item[\textit{(a.i)}] \textit{Probar que si $\varphi$ es una solución de $(H)$ en $\R$ con $\varphi(0)=\varphi(\omega)$, entonces $\varphi$ es $\omega$-periódica.}
        \item[\textit{(a.ii)}] \textit{Supongamos que $\Phi \colon \R \to \mathcal{M}_n(\R)$ es una matriz fundamental de $(H)$. ¿Qué significa eso? ¿Y que sea matriz fundamental canónica de $(H)$ en $0$? Si $\Phi$ no es matriz fundamental canónica de $(H)$ en 0, dar una función matricial $\Psi$ que sí lo sea (en términos de $\Phi$). Expresar el conjunto de soluciones de $(H)$ en términos de $\Phi$.}
        \item[\textit{(a.iii)}] \textit{Supongamos que $\Phi$ es una matriz fundamental canónica de $(H)$ en $0$. Probar que $(H)$ tiene solución $\omega$-periódica no trivial si y solo si un autovalor de $\Phi(\omega)$ es $\lambda=1$.}
    \end{itemize}
    \item[\textit{(b)}] \textit{Sean $p$ y $q$ constantes reales. Consideremos la ecuación $(E) \ y''+py'+qy=0$. Probar que son equivalentes:}
    \begin{itemize}
        \item[\textit{(b.i)}] $p=0$, $q>0$.
        \item[\textit{(b.ii)}] \textit{Todas las soluciones no triviales de $(E)$ son periódicas.}
    \end{itemize}
    \textit{Determinar en este caso el periodo minimal de las soluciones no triviales de $(E)$.}
\end{itemize}

\vspace{2mm}

\hrule

\vspace{2mm}

\begin{itemize}
    \item[\textit{(a)}] Sea $\varphi \colon \R \to \R^n$ una solución de $(H)$ verificando $\varphi(0)=\varphi(\omega)$. Sea $x^0=\varphi(0)=\varphi(\omega)$. Como $(S)$ es un sistema diferencial lineal de primer orden, el problema
    \[(P) \begin{cases}
        x'=A(t)x+b(t) \\
        x(0)=x^0
    \end{cases}\]
    tiene solución única en $\R$. Por una parte, tenemos que $\varphi$ es solución de $(P)$. Por otra parte, si se define $\varphi_\omega \colon \R \to \R^n$ mediante $\varphi_\omega(t) = \varphi(t+\omega)$, se tiene que
    \begin{itemize}
        \item[\textit{(i)}] $\varphi_\omega$ es derivable por serlo $\varphi$.
        \item[\textit{(ii)}] $\textup{gráf}(\varphi_{\omega}) \subset \R^2$, evidentemente.
        \item[\textit{(iii)}] Por la regla de la cadena, \[\varphi_\omega'(t) = \varphi'(t+\omega)=A(t+\omega)\varphi(t+\omega)=A(t)\varphi(t+\omega)=A(t)\varphi_\omega(t),\] donde se ha usado que $A$ es $\omega$-periódica.
        \item[\textit{(iv)}] $\varphi_\omega(0) = \varphi(\omega)=x^0$.
    \end{itemize}
    Por tanto, $\varphi_\omega$ es también solución de $(P)$, así que debe ser $\varphi_\omega = \varphi$ por asuntos de unicidad, concluyéndose que $\varphi$ es periódica de periodo $\omega$.

    \vspace{2mm}

Sea $\Phi$ una matriz fundamental de $(H)$. Esto significa que sus columnas forman una base del espacio de las soluciones de $(H)$, que es un subespacio vectorial de $\mathcal{C}^1(\R,\R)$. Que $\Phi$ sea matriz fundamental canónica de $(H)$ en 0 significa que $\Phi$ es matriz fundamental de $(H)$ y, además, es la única que verifica $\Phi(0)=\textup{Id}$. Si $\Phi$ no fuese matriz fundamental de $\Psi$ en 0, entonces la función matricial dada por $\Psi(t)=\Phi(t)\Phi^{-1}(0)$ sí lo es. Por último, si $\varphi \colon \R \to \R^n$ es una solución de $(H)$, entonces llamando $x^0=\varphi(0)$, se tiene que
\[\varphi(t)=\Phi(t)\Phi^{-1}(0)x^0\]

\vspace{2mm}

Por otra parte, sea $\Phi$ una matriz fundamental canónica de $(H)$ en 0, y sea $\varphi \colon \R \to \R^n$ es solución $\omega$-periódica de $(H)$ no trivial. Sea $x^0=\varphi(0)$ y considérese el problema
    \[(P) \begin{cases}
        x'=A(t)x \\
        x(0)=x^0
    \end{cases}\]
    Como $\Phi$ es matriz fundamental canónica en 0, la única solución de $(P)$ en $\R$ viene dada por
    \[\psi(t)=\Phi(t)x^0,\]
    y como $\varphi$ es solución de $(P)$, entonces
    \[\varphi(t)=\Phi(t)x^0\]
    Pero, por lo probado anteriormente, $\varphi(0)=\varphi(\omega)$, luego
    \[x^0=\Phi(\omega)x^0\]
    Equivalentemente,
    \[(\Phi(\omega)-\textup{Id})x^0=0,\]
    Obsérvese que $x^0 \neq 0$ (si fuese $x^0$ se contradiría que $\varphi$ es no trivial, ya que la función nula resolvería $(P)$), luego $x^0$ es un autovector de $\Phi(\omega)$ asociado al autovalor $\lambda=1$.

    \vspace{2mm}

    Recíprocamente, supóngase que existe $x^0 \neq 0$ tal que $\Phi(\omega)x^0=x^0$. Sea $\varphi \colon I \to \R^n$ la función definida por $\varphi(t) = \Phi(t) x^0$. Se tiene que
    \begin{itemize}
        \item[\textit{(i)}] $\varphi$ es solución de $(E)$, pues $\varphi'(t) = \Phi'(t)x^0 = A(t)\Phi(t)x^0=A(t)\varphi(t)$ para cada $t \in \R$.
        \item[\textit{(ii)}] $\varphi$ no es la función nula, pues $x^0 \neq 0$ y $\Phi$ es una matriz regular.
        \item[\textit{(iii)}] $\varphi(0) = \Phi(0)x^0=x^0=\Phi(\omega)x^0=\varphi(\omega)$, así que $\varphi$ es una función $\omega$-periódica.
    \end{itemize}
\item[\textit{(b)}] Se va a resolver $(E)$. La ecuación característica de $(E)$ es $\lambda^2+p\lambda+q=0$, y sus raíces,
\[\lambda_1=\frac{-p+\sqrt{p^2-4q}}{2}, \qquad \lambda_2=\frac{-p-\sqrt{p^2-4q}}{2}\]
Se distinguen los siguientes casos:
\begin{itemize}
    \item[\textit{(i)}] $p^2-4q>0$. Entonces $\lambda_1, \lambda_2 \in \R$ y $m(\lambda_1)=m(\lambda_2)=1$, luego un sistema fundamental de soluciones de $(E)$ es $\mathcal{F}=\{e^{\lambda_1t},e^{\lambda_2t}\}$. Por tanto, cualquier solución de $(E)$ es de la forma
    \[\varphi(t)=c_1e^{\lambda_1t}+c_2e^{\lambda_2t}, \quad c_1,c_2 \in \R\]
    \item[\textit{(ii)}] $p^2-4q<0$. Entonces $\lambda_1, \lambda_2 \in \mathbb{C}$ y $m(\lambda_1)=m(\lambda_2)=1$. La parte real de cada raíz es 
    \[\alpha=-\frac{p}{2},\] y la imaginaria, \[\beta=\pm \frac{\sqrt{4q-p^2}}{2},\] luego un sistema fundamental de soluciones de $(E)$ es $\mathcal{F}=\{e^{\alpha t}\cos(\beta t),e^{\alpha t}\sen(\beta t)\}$. Por tanto, cualquier solución de $(E)$ es de la forma
    \[\varphi(t)=c_1e^{\alpha t}\cos(\beta t)+c_2e^{\alpha t}\sen(\beta t), \quad c_1,c_2 \in \R\]
    \item[\textit{(iii)}] $p^2-4q=0$. Entonces $\lambda_1=\lambda_2=-\frac{p}{2}$, luego un sistema fundamental de soluciones de $(E)$ es
    \[\mathcal{F}=\{e^{-\frac{p}{2}t},te^{-\frac{p}{2}t}\}\]
    Por tanto, cualquier solución de $(E)$ es de la forma
    \[\varphi(t)=c_1e^{-\frac{p}{2}t}+c_2te^{-\frac{p}{2}t}, \quad c_1,c_2 \in \R\]
\end{itemize}
Como la exponencial no es una función periódica, en los casos $(i)$ y $(iii)$ no puede obtenerse ninguna solución no trivial y periódica. Por tanto, la única posibilidad de obtener soluciones periódicas no triviales se encuentra en el caso $(ii)$; concretamente, cuando se tenga $\alpha=0$, es decir, cuando se tenga $p=0$. En estas circunstancias, todas las soluciones no triviales de $(E)$ serían periódicas.

\vspace{2mm}

Se concluye que todas las soluciones no triviales de $(E)$ son periódicas si y solo si $p^2-4q<0$ y $p=0$, es decir, si y solo si $p=0$ y $q>0$. En ese caso, las soluciones en cuestión son de la forma
\[\varphi(t)=c_1\cos(\beta t)+c_2\sen(\beta t), \quad c_1,c_2 \in \R,\]
luego el periodo minimal es $\frac{2\pi}{\beta}=\frac{2\pi}{\sqrt{q}}$.

\end{itemize}

\vspace{2mm}

\hrule

\vspace{4mm}

\noindent 2. \textit{Resolver el siguiente problema de datos iniciales, justificando los cálculos con los resultados teóricos vistos en clase:}
\[(P) \begin{cases}
    y'''-y''+y'-y=5\sen(t)e^t \\
    y(0)=0, \ y'(0)=-3, \ y''(0)=-1
\end{cases}\]

\vspace{2mm}

\hrule

\vspace{4mm}

La ecuación $(E) \ y'''-y''+y'-y=5\sen(t)e^t$ es una ecuación diferencial lineal de coeficientes constantes de orden $3$. Por tanto, su solución general es $\varphi(t)=\varphi_h(t)+\varphi_p(t)$, donde $\varphi_h$ es la solución general de $(H) \ y'''-y''+y'-y=0$, y $\varphi_p$ es una solución particular de $(E)$.

\vspace{2mm}

Para hallar $\varphi_h$, se va a tratar de encontrar un sistema fundamental de soluciones de $(H)$. La ecuación característica asociada a $(H)$ es $\lambda^3-\lambda^2+\lambda-1=0$, o sea, $(\lambda-1)(\lambda^2+1)=0$. Los autovalores de $(H)$ son $\lambda_1=1$, $\lambda_2=i$ y $\lambda_3=-i$, luego un sistema fundamental de soluciones de $(H)$ es
\[\mathcal{F}=\{e^t,\cos(t),\sen(t)\},\]
luego
\[\varphi_h(t)=c_1e^t+c_2\cos(t)+c_3\sen(t), \qquad c_1,c_2,c_3 \in \R\]

Para el cálculo de $\varphi_p$, se empleará el método de los coeficientes indeterminados. Obsérvese que el término independiente de $(E)$ es de la forma
\[a(t)=e^{\alpha t}(q_1(t)\cos(\beta t)+q_2(t)\sen(\beta t)),\]
donde $\alpha=1$, $\beta =1$ y $q_1(t)=0$, $q_2(t)=5$ son polinomios de grado $0$. Por tanto, $(E)$ dispone de una solución particular de la forma
\[\varphi_p(t)=t^{m(\mu)}e^{\alpha t}(Q_1(t)\cos(\beta t)+Q_2(t)\sen(\beta t)),\]
donde $m(\mu)$ es la multiplicidad de $\mu=\alpha+i\beta = 1+i$ como autovalor de $(H)$ (o sea, $m(\mu) =0$), y $Q_1(t)=A$, $Q_2(t)=B$ son polinomios de grado $0$. En consecuencia,
\[\begin{aligned}[t]
    \varphi_p(t)&=Ae^t\cos(t)+Be^t\sen(t); \\
    \varphi_p'(t)&=(A+B)e^t\cos(t)+(B-A)e^t\sen(t); \\
    \varphi_p''(t)&= 2Be^t\cos(t)-2Ae^t\sen(t); \\
    \varphi_p'''(t)&=(2B-2A)e^t\cos(t)-(2A+2B)e^t\sen(t),
\end{aligned}\]
luego
\[
\begin{aligned}[t]
    \varphi_p \textup{ es solución de } (E) &\iff \varphi_p'''(t)-\varphi_p''(t)+\varphi_p'(t)-\varphi_p(t)=5e^t\sen(t) \\ 
    &\iff (B-2A)e^t\cos(t)-(A+2B)e^t\sen(t)=5e^t\sen(t) \\
    &\iff B-2A=0, \ A+2B=-5 \\
    &\iff A=-1, B=-2,
\end{aligned}
\]
así que $\varphi_p(t)=-e^t\cos(t)-2e^t\sen(t)$ es solución de $(E)$, así que la solución general de $(E)$ sería
\[\varphi(t)=c_1e^t+(c_2-e^t)\cos(t)+(c_3-2e^t)\sen(t), \quad c_1,c_2,c_3\in \R\]

Para resolver el problema dado, hay que encontrar el valor que deben tomar las constantes $c_1,c_2,c_3 \in \R$ para que se satisfagan los datos iniciales. Se tiene que

\[
\begin{aligned}[t]
    \varphi'(t)&=c_1e^t+(c_3-c_2-e^t)\cos(t)+(-c_2-2c_3+5e^t)\sen(t); \\
    \varphi''(t)&=c_1e^t+(-3c_3+6e^t)\cos(t)+(-4c_2-11c_3+26e^t)\sen(t), \\
\end{aligned}
\]
luego
\[
\begin{aligned}[t]
    \varphi(0)&=c_1+c_2-1=0; \\
    \varphi'(0)&=c_1+c_3-c_2-1=-3; \\
    \varphi''(0)&=c_1-3c_3+6=-1, \\
\end{aligned}
\]
de donde se deduce que $c_1=-10/7 ,c_2=17/7 $ y $c_3 =13/7$. A la vista de estas cifras tan feas, es altamente probable que existan errores de cálculo.

\vspace{4mm}

\hrule

\vspace{4mm}

\noindent 3. \textit{Realizar un estudio, lo más exhaustivo posible, de las soluciones no prolongables de la ecuación}
\[(E) \quad x'=t\sqrt{1-t^2-x^2}\]
\textit{¿En qué región puede asegurarse que la ecuación $(E)$ satisface la PUG porque se dan las hipótesis del TUG? El estudio debe incluir: datos iniciales posibles, intervalos maximales de definición de soluciones, comportamiento en los extremos de estos intervalos, monotonía de las soluciones, simetría, soluciones constantes (si las hay)...}

\vspace{4mm}

\hrule

\vspace{4mm}

En primer lugar, obsérvese que la función dada por $f(t,x) = t\sqrt{1-t^2-x^2}$ está bien definida en el conjunto
\[D=\{(t,x) \in \R^2 \colon t^2+x^2\leq 1\} = \overline{B}((0,0);1),\]
considerando como norma en $\R^2$ la norma euclídea. Además, si $(t,x) \in \mathring{D}$, entonces
\[\frac{\partial f}{\partial x} (t,x) = -\frac{tx}{\sqrt{1-t^2-x^2}},\]
que es una función continua en $\mathring{D}$. Por tanto, $f \in \mathcal{C}(\mathring{D},\R) \cap \textup{Lip}_{\textup{loc}}(x,\mathring{D},\R)$, así que, al verificarse las hipótesis del TUG, puede asegurarse que $(E)$ satisface la PUG en $\mathring{D}$. Sin embargo, si tomamos $(t,x) \in \partial D$ y $(t,y) \in \mathring{D}$, se tiene que
\[\frac{|f(t,x)-f(t,y)|}{|x-y|}=\frac{|t|\sqrt{1-t^2-y^2}}{|x-y|},\]
que tiene límite infinito cuando $y \to x$. Por tanto, $f$ no es de Lipschitz en ningún entorno de cualquer punto de la frontera de $D$, así que las hipótesis del TUG no se verifican en $D$; solo en su interior.

\vspace{2mm}

Consecuentemente, para cada dato inicial $(t_0,x_0) \in \mathring{D}$, puede asegurarse que el problema de Cauchy
\[(P) \begin{cases}
    x'=t\sqrt{1-t^2-x^2}\\
    x(t_0)=x_0
\end{cases}\]
posee solución local única (por el TEUL) con gráfica en $\mathring{D}$, que puede extenderse a una solución maximal $\varphi \colon I \to \R$ con gráfica en $\mathring{D}$, que además es única por satisfacerse la PUG en la región $\mathring{D}$. Además, como $\mathring{D}$ es abierto, entonces, por el resultado sobre soluciones maximales con gráficas en abiertos, se tiene que $I=(a,b)$, donde $-1\leq a <t_0<b\leq 1$. Además, por el mismo resultado, como los extremos de $I$ son finitos, si $t^*$ es un extremo de $I$, entonces se verifica una de las siguientes circunstancias:
\begin{itemize}
    \item[\textit{(i)}] $\lim_{t \to t^*}|\varphi(t)|=\infty$.
    \item[\textit{(ii)}] La gráfica de $\varphi$ tiene un punto límite para $t \to t^*$, y este y todos los puntos límite de la gráfica de $\varphi$ para $t \to t^*$ se hallan en $\partial D$.
\end{itemize}

Por otra parte, considérese la función $\psi \colon (-b,-a) \to \R$ definida mediante $\psi(t)=\varphi(-t)$. Se tiene que, por la regla de la cadena, $\psi$ es derivable y, para todo $t \in (-b,-a)$,
\[\psi'(t)=-\varphi'(-t)=t\sqrt{1-(-t)^2-\varphi(-t)^2}=t\sqrt{1-t^2-\psi(t)^2},\]
luego $\psi(t)$ es también solución de $(E)$ con gráfica en $\mathring{D}$. Pero $\varphi$ es la única solución maximal de $(E)$ con gráfica en $\mathring{D}$, así $\varphi$ debe ser una prolongación de $\psi$, es decir, $(a,b)\subset (-b,-a)$ y $\varphi |_{(-b,-a)}=\psi$. Ahora bien, que sea $(a,b) \subset (-b,-a)$ implica $a=-b$, luego, en realidad, $\varphi=\psi$, y por tanto $\varphi$ es una función par, lo que conlleva $I=(-b,b)$.

\vspace{2mm}

Por otra parte, obsérvese que, por ser $t^2+\varphi(t)^2<1$ para todo $t \in (-b,b)$, se tiene que $\varphi'(t)<0$ para todo $t \in (-b,0)$ y $\varphi'(t)>0$ para todo $t \in (0,b)$, así que $\varphi$ es estrictamente decreciente en $(-b,0]$ y estrictamente creciente en $[0,b)$.

\vspace{2mm}

También puede observarse que $f$ es acotada en $D$, así que, por ser los extremos de $I$ finitos, el resultado sobre soluciones con derivada acotada permite afirmar que existen $A=\lim_{t\to -b^+}\varphi(t)$ y $B = \lim_{t \to b^-}\varphi(t)$. Al unir esto con lo obtenido mediante el resultado sobre soluciones maximales con gráficas en abiertos, se puede descartar el caso $(i)$. En consecuencia, los puntos límite de la gráfica de $\varphi$ para $t \to -b$ son de la forma $(-b,\pm\sqrt{1-b^2})$, y para $t \to b$, de la forma $(b,\pm\sqrt{1-b^2})$. Pero sabemos que $\varphi$ tiene límite en $t^*$, así que los únicos posibles puntos límite son $(-b,A)$ y $(b,B)$, deduciéndose que, o bien $A=B=\sqrt{1-b^2}$, o bien $A=B=-\sqrt{1-b^2}$. Todo esto nos lleva a considerar la función $\widetilde{\varphi} \colon [-b,b] \to \R$ definida mediante
\[\widetilde{\varphi}(t)=\begin{cases}
    B &  $ si $ t \in \{-b,b\} \\
    \varphi(t) & $ en otro caso$
\end{cases}\]
y tenemos que $\widetilde{\varphi}$ es una prolongación estricta de $\varphi$ como solución de $(E)$. Obsérvese que esto no contradice la maximalidad de $\varphi$ como solución de $(E)$ con gráfica en $\mathring{D}$, pues la gráfica de $\widetilde{\varphi}$ se sale de $D$.

\vspace{2mm}

Por último, nótese que la ecuación $(E)$ no posee soluciones constantes. Supongo que con toda esta cantinela se dan por satisfechos los requerimentos del ejercicio.

\vspace{4mm}

\hrule

\vspace{4mm}

\noindent 4. \textit{Realizar un estudio, lo más exhaustivo posible, de las soluciones no prolongables de la ecuación}
\[(E) \quad x'=1-e^{1-x^2}\]
\textit{En este estudio debe probarse asimismo que el intervalo maximal de definición de cualquier solución no prolongable de $(E)$ es $\R$ (para ello, quizás haga falta recurrir a algún resultado sobre soluciones con derivada acotada). Esbozar el aspecto de las gráficas de estas posibles soluciones.}

\vspace{4mm}

\hrule

\vspace{4mm}

Sea $g \colon \R \to \R$ la función definida por $g(x)=1-e^{1-x^2}$. La ecuación $(E) \ x'=g(x)$ es una ecuación diferencial escalar autónoma de primer orden. Se tiene que
\[1-e^{1-x^2}=0 \iff e^{1-x^2}=1 \iff 1-x^2=0 \iff x=1,x=-1\]

Por tanto, $\varphi_{-1} \equiv -1$ y $\varphi_1\equiv 1$ son las únicas soluciones constantes en $\R$ de la ecuación $(E)$. Como $(E)$ verifica la PUG en $\R$ (pues $g \in \mathcal{C}^1(\R,\R)$), entonces la gráfica de cualquier solución maximal no constante no debe cortar a la gráfica de ninguna solución constante. En otras palabras, si $\varphi \colon I \to \R$ es una solución maximal de $(E)$ y consideramos las regiones
\[D_1= \R \times(-\infty,-1), \qquad D_2=\R \times (-1,1) \qquad \textup{y} \qquad D_3=\R \times (1,\infty),\]
entonces $\textup{gráf}(\varphi) \subset D_i$ para algún $i \in \{1,2,3\}$. Además, el resultado sobre soluciones maximales con gráficas en abiertos (sí, $\R^2$ es abierto) permite asegurar que $I=(a,b)$, y si $t^*$ es un extremo finito de $I$, entonces $\lim_{t \to t^*} |\varphi(t)|=\infty$ (la casuística de los puntos límite es imposible por tener $\R^2$ frontera vacía). Se distinguen tres casos:
\begin{itemize}
    \item[\textit{(i)}] $\textup{gráf}(\varphi) \subset D_1$. Entonces $\varphi(t)<-1$, así que $1-\varphi(t)^2<0$, y se verifica $\varphi'(t)=1-e^{1-\varphi(t)^2}>0$ para todo $t \in I$, obteniéndose que $\varphi$ es estrictamente creciente. Por tanto, existen $A=\lim_{t \to a^+}\varphi(t)$ y $B=\lim_{t \to b^{-}}\varphi(t)$. Si fuese $b<\infty$, entonces $\lim_{t \to b^{-}} |\varphi(t)|=\lim_{t \to b^{-}} -\varphi(t)=\infty$, luego $B=-\infty$, que es imposible por ser $\varphi$ estrictamente creciente. Por tanto, $b=\infty$, y como no puede ser $B=-\infty$, entonces $B=-1$ (si fuera $-\infty<B<-1$, se obtendría una nueva solución constante de $(E)$). En el otro extremo, en principio, pudiera ocurrir $a>-\infty$ o $a=-\infty$, pero en cualquier caso, $A=-\infty$ ($A=-1$ no puede ser por el crecimiento de $\varphi$; tampoco puede ser $-\infty<A<-1$ porque se obtendría otra solución constante). El resumen de este caso es que, o bien
    \[a>-\infty, \qquad A=-\infty, \qquad b=\infty \qquad \textup{y} \qquad B=-1,\]
    o bien
    \[a=-\infty, \qquad A=-\infty, \qquad b=\infty \qquad \textup{y} \qquad B=-1,\]
    \item[\textit{(ii)}] $\textup{gráf}(\varphi) \subset D_2$. Resulta que la gráfica de $\varphi$ queda encerrada entre la gráfica de dos soluciones constantes, así que ha de ser $I=\R$. Además, como $-1<\varphi(t)<1$, entonces $1-\varphi(t)^2 >0$, y, por tanto, $\varphi'(t)=1-e^{1-\varphi(t)^2}<0$ para todo $t \in \R$, así que $\varphi$ decrece estrictamente. De esto se deduce que $A=1$ y que $B=-1$. El resumen de este caso es
    \[a=-\infty, \qquad A=1, \qquad b=\infty \qquad \textup{y} \qquad B=-1,\]
    \item[\textit{(iii)}] $\textup{gráf}(\varphi) \subset D_3$. Entonces $\varphi(t)>1$, así que $1-\varphi(t)^2<0$, y se verifica $\varphi'(t)=1-e^{1-\varphi(t)^2}>0$ para todo $t \in I$, obteniéndose que $\varphi$ es estrictamente creciente. Por tanto, existen $A=\lim_{t \to a^+}\varphi(t)$ y $B=\lim_{t \to b^{-}}\varphi(t)$. Si fuese $a>-\infty$, entonces $\lim_{t \to a^{+}} |\varphi(t)|=\lim_{t \to a^{+}} \varphi(t)=\infty$, luego $A=\infty$, que es imposible por ser $\varphi$ estrictamente creciente. Por tanto, $a=-\infty$, y como no puede ser $A=\infty$, entonces $A=1$ (si fuera $1<A<\infty$, se obtendría una nueva solución constante de $(E)$). En el otro extremo, en principio, pudiera ocurrir $b<\infty$ o $b=\infty$, pero en cualquier caso, $B=\infty$ ($B=1$ no puede ser por el crecimiento de $\varphi$; tampoco puede ser $1<B<\infty$ porque se obtendría otra solución constante). El resumen de este caso es que, o bien
    \[a=-\infty, \qquad A=1, \qquad b=\infty \qquad \textup{y} \qquad B=\infty,\]
    o bien
    \[a=-\infty, \qquad A=1, \qquad b<\infty \qquad \textup{y} \qquad B=\infty,\]
\end{itemize}

Por último, vamos a adentrarnos en el caso $(i)$ para descartar la posibilidad de $a>-\infty$. Por reducción al absurdo, supóngase que $a>-\infty$, y sea $t_0 \in I$. Como $g=\varphi'$ es acotada en $(a,t_0]$, el resultado sobre soluciones con derivada acotada diría que $A = \lim_{t \to a^+} \in \R$, y esto contradice que $A=-\infty$, como se había visto en $(i)$. Por tanto, no puede ser $a>-\infty$, así que, en el caso $(i)$, $I=\R$. En el caso $(ii)$ ya se tiene $I=\R$, y en $(iii)$, se demuestra de forma totalmente análoga que ha de ser $I=\R$.

\end{document}