\documentclass[a4paper, 12pt, extrafontsizes]{memoir}

\usepackage{preamble}

\begin{document}

\pagestyle{empty}

\begin{center}

    \textbf{Examen de Ecuaciones en Derivadas Parciales y Análisis de Fourier}

    \textbf{16 de enero de 2024}

\end{center}

\begin{exercise}
\hfill
\begin{enumerate}
    \item Halla las curvas características de la ecuación
    \[u_x+2xu_y = 0, \qquad (x,y) \in \R^2.\]
    Representa las curvas características base. Comprueba que dos cualesquiera no se cortan en ningún punto. ¿Qué propiedad de la ecuación nos permite saber eso de antemano? Explica tu respuesta.
    \item Sea $h \in \mathcal{C}^1(\R)$. Justifica que para todo $x_0 \in \R$ con $x_0 \neq 0$, el problema
    \[\left\{\begin{alignedat}{3}
        u_x+2xu_y &= 0, \qquad & (x,y) &\in \R^2, \\
        u(x,0) &= h(x), \qquad & x &\in \R,
    \end{alignedat}\right.\]
    tiene solución única en un entorno de $(x_0,0)$, y hállala mediante el método de las características (especifica el dominio de la solución que obtienes). Estudia qué ocurre en $(0,0)$.
\end{enumerate}
\end{exercise}

\begin{exercise}
Consideramos el problema
\[\left\{\begin{alignedat}{3}
    u_{tt}(x,t)-c^2u_{xx}(x,t) &= 0, \qquad & x &\in \R, \ t > 0, \\
    u(x,0) &= 0, \qquad & x &\in \R, \\
    u_t(x,0) &= \varphi(x), \qquad & x &\in \R,
\end{alignedat}\right.\]
donde
\[\varphi(x) = \begin{cases}
    \sen^2(x) & $ si $ |x| \leq \pi, \\
    0 & $ si $ |x| > \pi.
\end{cases}\]
\begin{enumerate}
    \item ¿Cómo se interpreta ese problema en el modelo de la cuerda vibrante? Justifica que el problema tiene solución única y escribe la solución. Di qué regularidad tiene la solución.
    \item Supongamos $c = 1$. Calcula $u(2\pi,2\pi)$. Dibuja aproximadamente la gráfica de $u(2\pi,\cdot)$. Explica qué significa esa gráfica en el modelo de la cuerda vibrante.
\end{enumerate}
\end{exercise}

\begin{exercise}
Sea $f \colon [0,\pi] \to \R$, $f(x) = x(x-\pi)$.
\begin{enumerate}
    \item Halla la serie de Fourier de senos de $f$ y estudia la convergencia de la serie.
    \item Usa lo obtenido en el apartado $(a)$ para calcular la suma de la serie $\sum_{m=1}^\infty \frac{(-1)^m}{(2m-1)^3}$.
\end{enumerate}
Enuncia los teoremas que uses.
\end{exercise}

\begin{exercise}
\hfill
\begin{enumerate}
    \item Sea $\alpha \in \R$ con $0 < \alpha \leq 1$. Aplica el método de separación de variables al problema
    \[\left\{\begin{alignedat}{4}
    u_{t}(x,t)-u_{xx}(x,t) -\alpha u &= 0, \qquad & x &\in (0,\pi), \ & t > 0, \\
    u(0,t) &= 0, \qquad & & & t \geq 0, \\
    u(\pi,t) &= 0, \qquad & & & t \geq 0, \\
    u(x,0) &= u_0(x), \qquad & x &\in [0,\pi]. \\
    \end{alignedat}\right.\]
    \item Halla la solución formal que se obtiene cuando $u_0(x) = x(\pi-x)$.
    \item Justifica que la solución formal es realmente solución.
    \item Estudia el $\lim_{t \to \infty} u(x,t)$.
\end{enumerate}
\end{exercise}

\begin{exercise}
\hfill
\begin{enumerate}
    \item ¿Cómo se define la transformada de Fourier, $\widehat{f}$, de una función $f \in L^1(\R)$? Demuestra que $\widehat{f}$ es acotada y continua.
    \item Sea $f \colon \R \to \R$,
    \[f(x) = \begin{cases}
        -e^x & $ si $ x < 0, \\
        e^{-x} & $ si $ x \geq 0.
    \end{cases}\]
    Calcula $\widehat{f}$. Enuncia el teorema de inversión para la transformada de Fourier. ¿Cumple las hipótesis la $f$ anterior?
\end{enumerate}
\end{exercise}

\begin{exercise}
\hfill
\begin{enumerate}
    \item Sea $\Omega \subset \R^n$ un dominio acotado. Demostra el principio del máximo débil para el laplaciano. Úsalo para probar la unicidad de soluciones del problema de Dirichlet para la ecuación de Laplace en $\Omega$.
    \item Sea $D = \{x \in \R^n \colon \| x \| > 1\}$. Sea $u \in \mathcal{C}^2(D) \cap \mathcal{C}(\overline{D})$ tal que $\Delta u(x) = 0$ para todo $x \in D$ y $\lim_{\| x \| \to \infty} u(x) = 0$. Demuestra que
    \[\max_{x \in \overline{D}} |u(x)| = \max_{x \in \partial D} |u(x)|.\]
\end{enumerate}
\end{exercise}

\newpage

\begin{center}

    \textbf{Resolución}

\end{center}

\begin{solution}
\hfill
\begin{enumerate}
    \item La ecuación del enunciado es una ecuación del estilo
    \[b_1(x,y,u)u_x+b_2(x,y,u)u_y = c(x,y,u), \qquad (x,y) \in \R^2,\]
    donde $b_1(x,y,u) = 1$, $b_2(x,y,u) = 2x$ y $c(x,y,u) = 0$. Por tanto, el sistema característico es
    \[\left\{\begin{alignedat}{3}
        \varphi_1'(t) &= 1, \qquad & t &\in \R, \\
        \varphi_2'(t) &= 2\varphi_1(t), \qquad & t &\in \R, \\
        \eta(t) &= 0, \qquad & t &\in \R.
    \end{alignedat}\right.\]
    Las soluciones del sistema son las funciones $(\varphi_1,\varphi_2,\eta) \colon \R \to \R^3$ dadas por
    \[\varphi_1(t) = t+A, \qquad \varphi_2(t) =  t^2+2At+B, \qquad \eta(t) = C,\]
    con $A,B,C \in \R$ constantes cualesquiera. Hay que representar el conjunto
    \[\bigl\{\bigl(t+A, t^2+2At+B\bigr) \colon A,B, t \in \R\bigr\} = \bigl\{\bigl(s, (s-A)^2+2A(s-A)+B\bigr) \colon A,B, s \in \R\bigr\},\]
    que una unión de parábolas que pasan por $(0,B-A^2)$.
    \[\textup{INSÉRTESE AQUÍ LA REPRESENTACIÓN GRÁFICA EN CUESTIÓN}\]
    Se observa que dos características base cualesquiera no es cortan en ningún punto. Esto es porque el sistema característico es un sistema de ecuaciones diferenciales ordinarias lineales de primer orden, y un sistema de este tipo, fijado un dato inicial $(x_0,y_0,z_0)$ en $t = 0$, tiene solución única. 
    \item El problema del enunciado se puede escribir como     
    \[\left\{\begin{alignedat}{3}
        u_x+2xu_y &= 0, \qquad & (x,y) &\in \R^2, \\
        u(\sigma(s)) &= \tau(s), \qquad & s &\in \R,
    \end{alignedat}\right.\]
    donde $\sigma(s) = (s,0)$ y $\tau(s) = h(s)$ para cada $s \in \R$. Para probar que el problema posee solución única en un entorno de $(x_0,0)$, basta comprobar que se verifica la condición de transversalidad. Se tiene que $(x_0,0) = \sigma(x_0)$, y como
    \[\det\left(\begin{array}{cc}
        \sigma_1'(x_0) & b_1(\sigma_1(x_0),\sigma_2(x_0),\tau(x_0)) \\
        \sigma_2'(x_0) & b_2(\sigma_1(x_0),\sigma_2(x_0),\tau(x_0)) \\
    \end{array}\right) = \det\left(\begin{array}{cc}
        1 & 1 \\
        0 & 2x_0 \\
    \end{array}\right) = 2x_0 \neq 0,\]
    entonces el problema tiene solución única en un entorno de $\sigma(x_0) = (x_0,0)$. Para hallarla mediante el método de las características, hay que encontrar la única curva característica que en $t = 0$ pasa por $(s,0,h(s))$, para cualquier $s \in \R \setminus \{0\}$. Se tiene que
    \begin{itemize}
        \item $\varphi_1(0) = s \iff A = s$;
        \item $\varphi_2(0) = 0 \iff B = 0$;
        \item $\eta(0) = h(s) \iff C = h(s)$. 
    \end{itemize}
    Sean
    \[\varphi_1(t,s) = t+s, \qquad \varphi_2(t,s) = t^2+2st, \qquad \eta(t,s) = h(s).\]
    Se trata de hallar la inversa de $\varphi = (\varphi_1,\varphi_2)$, allá donde esté definida. Se tiene que
    \begin{align*}
        \begin{cases}
            t +s = x \\
            t^2+2st = y
        \end{cases} &\iff \begin{cases}
            s = x-t \\
            t^2+2(x-t)t = y
        \end{cases} \\ &\iff \begin{cases}
            s = x-t \\
            t^2-2xt+y = 0
        \end{cases} \\ &\iff \begin{cases}
            s = x-t \\
            t = \frac{2x\pm \sqrt{4x^2-4y}}{2}
        \end{cases} \\ &\iff \begin{cases}
            s = \mp \sqrt{x^2-y} \\
            t = x \pm \sqrt{x^2-y}
        \end{cases}
    \end{align*}
    Sea $U=\bigl\{(x,y) \in \R^2 \colon y < x^2\bigr\}$. Entonces $\varphi$ tiene inversa en $U$, y la expresión que define a $\varphi^{-1}$ depende del dominio de $\varphi$. Se tiene que
    \begin{itemize}
        \item
        $\begin{aligned}[t]
            \bigl\{\bigl(\varphi_1(t,s),\varphi_2(t,s),\eta(t,s)\bigr) \colon t \in \R, s > 0\bigr\} &= \bigl\{\bigl(x,y,\eta (\varphi^{-1}(x,y))\bigr) \colon x,y \in U\bigr\} \\
            &= \bigl\{\bigl(x,y, h(\textstyle{\sqrt{x^2-y}})\bigr) \colon x,y \in U\bigr\}.
        \end{aligned}$
        \item
        $\begin{aligned}[t]
            \bigl\{\bigl(\varphi_1(t,s),\varphi_2(t,s),\eta(t,s)\bigr) \colon t \in \R, s < 0\bigr\} &= \bigl\{\bigl(x,y,\eta (\varphi^{-1}(x,y))\bigr) \colon x,y \in U\bigr\} \\
            &= \bigl\{\bigl(x,y, h(\textstyle{-\sqrt{x^2-y}})\bigr) \colon x,y \in U\bigr\}.
        \end{aligned}$
    \end{itemize}
    En consecuencia:
    \begin{itemize}
        \item Si $x_0 > 0$, entonces la función $u \colon U \to \R$ definida por $u(x,y) = h(\sqrt{x^2-y})$ es unión de curvas características y, por tanto, es solución de la ecuación del problema. También verifica $u(x,0) = h(x)$, así que es solución del problema.
        \item Y si $x_0 < 0$, entonces la función $u \colon U \to \R$ definida por $u(x,y) = h(-\sqrt{x^2-y})$ es unión de curvas características y, por tanto, es solución de la ecuación del problema. También verifica $u(x,0) = h(x)$, así que es solución del problema. 
    \end{itemize} 
    Se observa en cualquier caso que el problema tiene solución en $U$, que es un entorno de $(x_0,0)$, tal y como adelantaba la condición de transversalidad. Si $x_0 = 0$, no puede asegurarse que el problema tenga solución.
\end{enumerate}
\end{solution}

\newpage

\begin{solution}
\hfill
\begin{enumerate}
    \item En el modelo de la cuerda vibrante, si $u \colon \R \times[0,\infty) \to \R$ es solución del problema, entonces $u(x,t)$ es la altura del punto $x$ de la cuerda (la cuerda se identifica con $\R$) en el instante $t$. Que se tenga $u(x,0)=0$ para todo $x \in \R$ quiere decir que en el instante inicial la cuerda se encuentra en posición horizontal, y que se tenga $u_t(x,0) = \varphi(x)$ para todo $x \in \R$ quiere decir que la velocidad a la que se mueve el punto $x$ en el instante inicial es $\varphi(x)$. En particular, los puntos de la cuerda que verifiquen $|x| > \pi$ se encuentran en reposo en el instante inicial. 
    
    Veamos que $\varphi \in \mathcal{C}^1(\R)$. Es claro que $\varphi \in \mathcal{C}^1((-\infty,-\pi) \cup (-\pi,\pi) \cup (\pi,\infty))$, así que solo hay que probar que $\varphi$ es derivable en $-\pi$ y $\pi$ y que su derivada es continua en dichos puntos. Se tiene que
    \begin{itemize}
        \item $\displaystyle\lim_{x \to {-\pi^-}} \frac{\varphi(x)-\varphi(-\pi)}{x+\pi} = \displaystyle\lim_{x \to {-\pi^-}} \frac{0-\sen^2(-\pi)}{x+\pi}  = 0$;
        \item $\displaystyle\lim_{x \to {-\pi^+}} \frac{\varphi(x)-\varphi(-\pi)}{x+\pi} = \displaystyle\lim_{x \to {-\pi^+}} \frac{\sen^2(x)}{x+\pi} \overset{\textup{L'H}}{=} \lim_{x \to {-\pi^+}} 2\sen(x)\cos(x) = 0$;
    \end{itemize}
    luego $\varphi$ es derivable en $-\pi$ y $\varphi'(-\pi) = 0$. Como
    \begin{itemize}
        \item $\displaystyle\lim_{x \to -\pi^-}\varphi'(x) = \lim_{x \to -\pi^-}0 = 0 = \varphi'(-\pi)$;
        \item $\displaystyle\lim_{x \to -\pi^+}\varphi'(x) = \lim_{x \to -\pi^-}0 = 2\sen(x)\cos(x) = 0 = \varphi'(-\pi)$;
    \end{itemize}
    entonces $\varphi'$ es continua en $-\pi$. De forma totalmente análoga se prueba que $\varphi$ es derivable en $\pi$ y que $\varphi'$ es continua en $\pi$.

    En consecuencia, por la fórmula de D'Alembert, la única solución del problema es la función $u \colon \R \times [0,\infty) \to \R$ dada por
    \[u(x,t) = \frac{1}{2c}\int_{x-ct}^{x+ct} \varphi(s) \, ds.\]
    Como $u$ es solución de un problema inicial asociado a la ecuación de ondas, se tiene que $u \in \mathcal{C}^2(\R \times (0,\infty)) \cap \mathcal{C}^{0,1}(\R \times[0,\infty))$ (esto último quiere decir que $u$ es continua respecto de $x$ y de clase $1$ respecto de $t$ en $\R \times [0,\infty)$). De hecho, como $u_0 \equiv 0 \in \mathcal{C}^2(\R)$ y $u_1 \equiv \varphi \in \mathcal{C}^1(\R)$, entonces $u \in \mathcal{C}^2(\R \times [0,\infty))$.
    \item Usando el apartado anterior, se tiene
    \[u(2\pi,2\pi) = \frac{1}{2}\int_0^{4\pi}\varphi(s) \, ds = \frac{1}{2}\int_0^{\pi}\sen^2(s) \, ds = \frac{1}{4}\int_0^{\pi}(1-\cos(2s)) \, ds = \frac{\pi}{4}.\]
    Además, para todo $t > 0$,
    \[u(2\pi,t) = \frac{1}{2}\int_{2\pi-t}^{2\pi+t} \varphi(s) \, ds.\]
    Se distinguen los siguientes casos:
    \begin{itemize}
        \item Si $2\pi-t \leq -\pi$, es decir, si $t \geq 3\pi$, entonces
        \[u(2\pi,t) = \frac{1}{2}\int_{-\pi}^\pi \sen^2(s) \, ds = \frac{1}{4}\int_{-\pi}^\pi (1-\cos(2s)) \, ds = \frac{\pi}{2}.\]
        \item Si $-\pi < 2\pi-t < \pi$, es decir, si $\pi < t < 3\pi$, entonces
        \[\begin{aligned}[t]
            u(2\pi,t) &= \frac{1}{2}\int_{2\pi-t}^\pi \sen^2(s) \, ds = \frac{1}{4}\int_{2\pi-t}^\pi (1-\cos(2s)) \, ds = \frac{t-\pi}{4}-\frac{1}{4}\left[\frac{\sen(2s)}{2}\right]_{s=2\pi-t}^{s=\pi} \\
            &= \frac{t-\pi}{4} +\frac{\sen(4\pi-2t)}{8}.
        \end{aligned}\]
        \item Si $\pi \leq 2\pi -t$, es decir, si $0 < t \leq \pi$, entonces 
        \[u(2\pi,t) = 0.\]
    \end{itemize}
    \[\textup{INSÉRTESE AQUÍ LA REPRESENTACIÓN GRÁFICA EN CUESTIÓN}\]
    En el modelo de la cuerda vibrante, esto quiere decir que antes del instante $\pi$, el punto $2\pi$ de la cuerda permanece en reposo, y que a partir del instante $3\pi$, dicho punto se encuentra a altura constate $\frac{\pi}{2}$.
\end{enumerate}
\end{solution}

\begin{solution}
    Es claro que $f \in L^1([0,\pi])$, así que tiene sentido considerar la serie de Fourier de senos de $f$. Los coeficientes de Fourier de senos de $f$ son, para cada $k \in \N$,
    \[
        b_k(f) = \frac{2}{\pi}\int_0^\pi f(x)\sen(kx) \, dx = \frac{2}{\pi}\int_0^\pi x^2\sen(kx) \, dx -2\int_0^\pi x\sen(kx) \, dx \tag{$\ast$}
    \]
    Integrando por partes,
    \[\begin{aligned}[t]
        \int_0^\pi x^2\sen(kx) \, dx &= \left[-\frac{x^2\cos(kx)}{k}\right]_{x=0}^{x=\pi}-\int_0^\pi -\frac{2x\cos(kx)}{k}\, dx \\
        &= -\frac{\pi^2(-1)^k}{k} +\frac{2}{k}\int_0^\pi x\cos(kx) dx \\
        &= -\frac{\pi^2(-1)^k}{k} +\frac{2}{k}\left(\left[\frac{x\sen(kx)}{k}\right]_{x=0}^{x=\pi}-\frac{1}{k}\int_0^\pi \sen(kx) \, dx \right) \\
        &= -\frac{\pi^2(-1)^k}{k} -\frac{2}{k^2}\left[-\frac{\cos(kx)}{k}\right]_{x=0}^{x=\pi}\\
        &= -\frac{\pi^2(-1)^k}{k} +\frac{2}{k^3}((-1)^k-1),\\
    \end{aligned}\]
    mientras que
    \[\begin{aligned}[t]
        \int_0^\pi x\sen(kx) \, dx &= \left[-\frac{x\cos(kx)}{k}\right]_{x=0}^{x=\pi} +\frac{1}{k}\int_0^\pi \cos(kx) \, dx = -\frac{\pi(-1)^k}{k}.
    \end{aligned}\]
    Llevando todo esto a $(\ast)$,
    \[\begin{aligned}[t]
        b_k(f) = \cancel{-\frac{2\pi(-1)^k}{k}}+\frac{4}{\pi k^3}((-1)^k-1) +\cancel{\frac{2\pi(-1)^k}{k}} = \begin{cases}
            0 & $ si $ k = 2m, \ m \in \N, \\
            -\frac{8}{\pi (2m-1)^3} & $ si $ k = 2m-1, \ m \in \N.
        \end{cases}
    \end{aligned}\]
    En consecuencia, la serie de Fourier de senos de $f$ es
    \[Sf(x) = \sum_{k=1}^\infty b_k(f) \sen(kx) = -\frac{8}{\pi}\sum_{m=1}^\infty \frac{1}{(2m-1)^3} \sen((2m-1)x), \qquad x \in [0,\pi].\]
    Como $\bigl\{\sen(k \cdot)\bigr\}_{k=1}^\infty$ es un sistema ortogonal completo en $L^2([0,\pi])$, entonces la serie de Fourier de $f$ converge a $f$ en $\|\cdot\|_2$. Y como $f$ es de clase $1$ (en particular, continua y de clase $1$ a trozos) y $f(0) = f(\pi) = 0$, entonces la serie de Fourier de senos de $f$ converge uniformemente a $f$ en $[0,\pi]$. Por tanto, también converge puntualmente a $f$ en $[0,\pi]$, es decir, $Sf(x) = f(x)$ para todo $x \in [-\pi,\pi]$. En particular, $Sf(\frac{\pi}{2}) =  f(\frac{\pi}{2})$. Se tiene que $f(\frac{\pi}{2}) = \frac{\pi}{2}(\frac{\pi}{2}-\pi) = -\frac{\pi^2}{4}$ y $Sf(\frac{\pi}{2}) = -\frac{8}{\pi}\sum_{m=1}^\infty \frac{(-1)^{m+1}}{(2m-1)^3}$, luego
    \[-\frac{\pi^2}{4} = \frac{8}{\pi}\sum_{m=1}^\infty \frac{(-1)^m}{(2m-1)^3},\]
    de donde 
    \[\sum_{m=1}^\infty \frac{(-1)^m}{(2m-1)^3} = -\frac{\pi^3}{32}.\]
    Se ha utilizado el resultado siguiente:
    \begin{theorem}
        Sea $f\in L^1([0,\pi])$ continua, de clase $1$ a trozos y tal que $f(0)=f(\pi)$. Sean $b_k(f)$, $k \in \N$, los coeficientes de Fourier de senos de $f$. Entonces
        \[\sum_{k=1}^\infty |b_k(f)| < \infty\]
        y la serie
        \[\sum_{k=1}^\infty b_k(f) \sen(kx), \qquad x \in [0,\pi], \]
        converge uniformemente a $f$ en $[0,\pi]$.
    \end{theorem}
\end{solution}

\begin{solution}
\hfill
\begin{enumerate}
    \item Se trata de hallar una solución no nula del problema del enunciado que sea de la forma
    \[w(x,t) = v(x)S(t), \qquad x \in (0,\pi), \ t > 0,\]
    y que verifique las condiciones en los extremos. Suponiendo que $S(t) \neq 0$ para todo $t > 0$ y $v(x) \neq 0$ para todo $x \in \R$, se tiene
    \[\begin{aligned}[t]
        w \textup{ resuelve la ecuación} &\iff w_t(x,t)-w_{xx}(x,t) -\alpha w(x,t) = 0 \ \forall \ (x,t) \in \R \times (0,\infty) \\
        &\iff S'(t)v(x)-S(t)v''(x)-\alpha S(t)v(x) = 0 \ \forall \ (x,t) \in \R \times (0,\infty) \\
        &\iff S'(t)v(x)-\alpha S(t)v(x)=S(t)v''(x) \ \forall \ (x,t) \in \R \times (0,\infty) \\
        &\iff \frac{S'(t)}{S(t)}-\alpha=\frac{v''(x)}{v(x)} \ \forall \ (x,t) \in \R \times (0,\infty) \\
        &\iff \ \exists \ \lambda \in \R \colon \lambda = \frac{S'(t)}{S(t)}-\alpha=\frac{v''(x)}{v(x)} \ \forall \ (x,t) \in \R \times (0,\infty)  \\
        &\iff \ \exists \ \lambda \in \R \colon \left\{\begin{alignedat}{3}
            S'(t) -(\alpha+\lambda)S(t) &= 0, \qquad &t > 0, \\
            v''(x) - \lambda v(x) &=0, \qquad &x \in \R,
        \end{alignedat}\right.
    \end{aligned}\]
    Nótese que independientemente de que $S$ o $v$ se anulen, sigue siendo cierto que si existe $\lambda \in \R$ tal que
    \[\left\{\begin{alignedat}{3}
        S'(t) -(\alpha+\lambda)S(t) &= 0, \qquad &t > 0, \\
        v''(x) - \lambda v(x) &=0, \qquad &x \in \R,
    \end{alignedat}\right.\]
    entonces $w$ es solución de la ecuación. Además, usando que $S$ no es idénticamente nula, se tiene que
    \[w(0,t) = w(\pi,t) = 0 \ \forall \ t \geq 0 \iff v(0) = v(\pi) =  0. \]
    Por tanto, hay que hallar $\lambda \in \R$ tal que $v$ sea solución no nula del problema
    \[\left\{\begin{alignedat}{3}
        v''(x) - \lambda v(x) &=0, \qquad &x \in \R, \\
        v(0) &= 0, \\
        v(\pi) &= 0,
    \end{alignedat}\right.\]
    y $S$, para el mismo $\lambda \in \R$, sea solución de la ecuación
    \[S'(t) -(\alpha+\lambda)S(t) = 0, \qquad t > 0.\]
    Se conoce que el problema para $v$ tiene soluciones no nulas si y solo si $\lambda \in \{\lambda_k \colon k \in \N\}$, donde $\lambda_k = -k^2$, y las soluciones de dicho problema son
    \[v_k(x) = \sen(kx) , \qquad x \in [0,\pi],\]
    o cualquier múltiplo. La ecuación para $S$ con $\lambda = \lambda_k$ es
    \[S_k'(t) =(\alpha-k^2)S_k(t), \qquad t > 0,\]
    que no es más que una sencilla ecuación diferencial ordinaria lineal de primer orden homogénea. Las soluciones de esta ecuación son
    \[S_{k}(t) = A_ke^{(\alpha-k^2)t}, \qquad t > 0,\]
    con $A_k \in \R$ una constante cualquiera. Así, la solución formal del problema sería
    \[u(x,t) = \sum_{k=1}^\infty A_ke^{(\alpha-k^2)t}\sen(kx), \qquad x \in \R, \ t > 0.\]
    Si $x \in \R$, se tiene que
    \[u(x,0) = u_0(x) \iff \sum_{k=1}^\infty A_k\sen(kx) = u_0(x),\]
    luego hay que tomar como $A_k$ los coeficientes de Fourier de $u_0$ en el sistema de senos de $[0,\pi]$. La solución formal del problema quedaría
    \[u(x,t) = \sum_{k=1}^\infty b_k(u_0)e^{(\alpha-k^2)t}\sen(kx), \qquad x \in \R, \ t > 0.\]
    \item Basta utilizar el apartado anterior y el ejercicio anterior para obtener que la solución formal del problema cuando $u_0(x) = x(\pi-x)$ viene dada por
    \[u(x,t) =-\frac{8}{\pi} \sum_{m=1}^\infty\frac{e^{(\alpha-(2m-1)^2)t}}{(2m-1)^3}\sen((2m-1)x), \qquad x \in \R, \ t > 0.\]
    \item Veamos primero que la serie que define a $u$ converge uniformemente en $[0,\pi] \times [0,\infty)$, lo que probará que $u$ está bien definida y es continua (cada sumando de la serie define una función continua, y el límite uniforme de una sucesión de funciones continuas es una función continua). Si $x \in \R$, $t \geq 0$ y $m \in \N$, se tiene que
    \[\biggl|\frac{8e^{(\alpha-(2m-1)^2)t}}{\pi(2m-1)^3}\sen((2m-1)x)\biggr| \leq |b_{2m-1}(u_0)|,\]
    donde se ha utilizado que $\alpha - (2m-1)^2 \leq 0$ para todo $m \in \N$ por ser $\alpha \leq 1$.
    Como $\sum_{k=1}^\infty |b_k(u_0)| < \infty$, el criterio de Weierstrass proporciona la convergencia uniforme de la serie en $[0,\pi] \times[0,\infty)$.

    Para probar que $u$ es realmente solución de la ecuación, basta probar que las series de las derivadas parciales respecto de $x$ (dos veces) y respecto de $t$ convergen uniformemente en $(0,\pi) \times (0,\infty)$, o lo que es lo mismo, que convergen uniformemente en $(0,\pi) \times [t_0,\infty)$, con $t_0 > 0$ fijo.
    \begin{itemize}
        \item Se tiene que
        \[\begin{aligned}
            \biggl|\frac{\partial}{\partial x}\left(b_{2m-1}(u_0)e^{(\alpha-(2m-1)^2)t}\sen((2m-1)x)\right)\biggr| &\leq |b_{2m-1}(u_0)|(2m-1)e^{(\alpha-(2m-1)^2)t_0}
        \end{aligned}\]
        para todos $x \in \R$, $t \in [t_0,\infty)$ y $m \in \N$. Como $|b_{k}(u_0)| \kconv 0$ (por el lema de Riemann-Lebesgue), existe $M > 0$ tal que $|b_{2m-1}(u_0)| \leq M$ para todo $m \in \N$. Además, para todo $\beta > 0$ se tiene que
        \[\frac{(k+1)^\beta e^{(\alpha-(k+1)^2)t_0}}{k^\beta e^{(\alpha-k^2)t_0}} = \left(1+\frac{1}{k}\right)^\beta e^{(\alpha-(k+1)^2-\alpha+k^2)t_0} = \left(1+\frac{1}{k}\right)^\beta e^{-(2k+1)t_0} \kconv 0,\]
        lo que permite afirmar, por el criterio del cociente, que $\sum_{k=1}^\infty k^\beta e^{(\alpha-k^2)t_0} < \infty$, y por tanto, que $\sum_{m=1}^\infty(2m-1)^\beta e^{(\alpha-(2m-1)^2)t_0} < \infty$. Tenemos entonces que
        \[\sum_{m=1}^\infty |b_{2m-1}(u_0)|(2m-1)e^{(\alpha-(2m-1)^2)t_0},\]
        luego, por el criterio de Weierstrass, la serie de las derivadas respecto de $x$ converge uniformemente en $(0,\pi) \times [t_0,\infty)$. Gracias a esto, $u$ tiene derivadas parciales respecto de $x$ en $(0,\pi) \times (0,\infty)$ y además son continuas (cada sumando de la serie de las derivadas define una función continua).
        \item El mismo argumento de antes sirve para probar que la serie
        \[\sum_{m=1}^\infty\frac{\partial^2}{\partial x^2}\left(b_{2m-1}(u_0)e^{(\alpha-(2m-1)^2)t}\sen((2m-1)x)\right)\]
        converge uniformemente en $(0,\pi) \times (0,\infty)$, lo que permite afirmar que $u$ es dos veces derivable respecto de $x$ en $(0,\pi) \times (0,\infty)$, y también que $u_{xx}$ es continua.
        \item  Se tiene que
        \[\begin{aligned}
            \biggl|\frac{\partial}{\partial t}\left(b_{2m-1}(u_0)e^{(\alpha-(2m-1)^2)t}\sen((2m-1)x)\right)\biggr| &\leq M(\alpha+(2m-1)^2)e^{(\alpha-(2m-1)^2)t_0}
        \end{aligned}\]
        para todos $x \in \R$, $t \in [t_0,\infty)$ y $m \in \N$. Como se ha probdo anteriormente que $\sum_{m=1}^\infty (2m-1)^\beta e^{(\alpha-(2m-1)^2)t_0} < \infty$ para todo $\beta \geq 0$, se tiene que
        \[\sum_{m=1}^\infty M(\alpha+(2m-1)^2)e^{(\alpha-(2m-1)^2)t_0} < \infty,\]
        así que, por el criterio de Weierstrass, la serie de las derivadas respecto de $t$ converge uniformemente en $(0,\pi) \times [t_0,\infty)$. Gracias a esto, $u$ tiene derivadas parciales respecto de $t$ en $(0,\pi) \times (0,\infty)$, y además son continuas (cada sumando de la serie de las derivadas define una función continua).
    \end{itemize}
    Con todo esto puede afirmarse que $u \in \mathcal{C}^2((0,\pi) \times (0,\infty)) \cap \mathcal{C}([0,\pi] \times [0,\infty))$. Que $u$ satisface la ecuación y las condiciones en los extremos es inmediato, pues por lo que se acaba de probar, la serie que define a $u$ puede derivarse término a término, y cada término es una solución de la ecuación que verifica las condiciones en los extremos. Por último, que $u$ satisface la condición inicial es cierto porque la serie de Fourier de senos de $u_0$ converge puntualmente (de hecho, converge uniformemente) a $u_0$ en $[0,\pi]$, tal y como se vio en el ejercicio anterior. Se concluye que $u$ es realmente solución del problema.
    \item Fijemos $x \in \R$ y veamos que
    \[\lim_{t \to \infty} u(x,t) = 0.\]
    Sea $\varepsilon > 0$. Hay que probar que existe $M > 0$ tal que para todo $t > M$ se tiene que $|u(x,t)| < \varepsilon$.
    \begin{itemize}
        \item Como la serie
        \[-\frac{8}{\pi} \sum_{m=1}^\infty\frac{e^{(\alpha-(2m-1)^2)t}}{(2m-1)^3}\sen((2m-1)x) \]
        converge uniformemente a $u$ en $[0,\infty)$ (se recuerda que $x$ está fijo), existe $n_0 \in \N$ tal que para todo $n \geq n_0$ se tiene que
        \[\biggl|u(x,t) +\frac{8}{\pi}\sum_{m=1}^n \frac{e^{(\alpha-(2m-1)^2)t}}{(2m-1)^3}\sen((2m-1)x) \biggr| < \frac{\varepsilon}{2}\]
        para todo $t \geq 0$. En particular,
        \[\biggl|u(x,t) +\frac{8}{\pi}\sum_{m=1}^{n_0} \frac{e^{(\alpha-(2m-1)^2)t}}{(2m-1)^3}\sen((2m-1)x) \biggr| < \frac{\varepsilon}{2}\]
        para todo $t \geq 0$, es decir,
        \[\biggl|\frac{8}{\pi}\sum_{m=n_0+1}^\infty \frac{e^{(\alpha-(2m-1)^2)t}}{(2m-1)^3}\sen((2m-1)x) \biggr| < \frac{\varepsilon}{2}\]
        para todo $t \geq 0$.
        \item Para todo $m \in \N$ se tiene que $\alpha-(2m-1)^2 < 0$ y por tanto
        \[\lim_{t \to \infty}\frac{8}{\pi}\frac{e^{(\alpha-(2m-1)^2)t}}{(2m-1)^3}\sen((2m-1)x) = 0,\]
        luego existe $M > 0$ tal que para todo $t > M$ se tiene que
        \[\biggl|\frac{8}{\pi}\frac{e^{(\alpha-(2m-1)^2)t}}{(2m-1)^3}\sen((2m-1)x)\biggr| < \frac{\varepsilon}{2n_0}.\]
        En consecuencia, para todo $t > M$,
        \[\sum_{m=1}^{n_0}\biggl|\frac{8}{\pi}\frac{e^{(\alpha-(2m-1)^2)t}}{(2m-1)^3}\sen((2m-1)x)\biggr| \leq \sum_{m=1}^{n_0} \frac{\varepsilon}{2n_0} = \frac{\varepsilon}{2}.\]
    \end{itemize}
    De esto se obtiene que para todo $t > M$,
    \[\begin{aligned}
        |u(x,t)| &= \biggl|\frac{8}{\pi}\sum_{m=1}^{n_0} \frac{e^{(\alpha-(2m-1)^2)t}}{(2m-1)^3}\sen((2m-1)x) + \frac{8}{\pi}\sum_{m=n_0+1}^\infty \frac{e^{(\alpha-(2m-1)^2)t}}{(2m-1)^3}\sen((2m-1)x)\biggr| \\
        &< \frac{\varepsilon}{2}+\frac{\varepsilon}{2} = \varepsilon.\\
    \end{aligned}\]
\end{enumerate}
\end{solution}

\begin{solution}
\hfill
\begin{enumerate}
    \item Si $f \in L^1(\R)$, la \emph{transformada de Fourier de $f$} es la función $\widehat{f} \colon \R \to \C$ dada por
    \[\widehat{f}(\xi)  = \frac{1}{\sqrt{2\pi}}\int_\R f(x)e^{-i\xi x} \, dx.\]
    Nótese que
    \[\int_\R |f(x)e^{-i\xi x}| \, dx = \|f\|_1 < \infty,\]
    así que la integral que define a $\widehat{f}$ tiene perfecto sentido. De esto también se obtiene que para todo $\xi \in \R$,
    \[|\widehat{f}(\xi)| \leq \frac{1}{2\pi}\int_\R|f(x)e^{-i\xi x}| \, dx = \frac{\|f\|_1}{\sqrt{2\pi}},\]
    luego $\widehat{f}$ es acotada. Veamos que $\widehat{f}$ es uniformemente continua, lo que probará que es continua. Sean $\{\xi_k\}_{k=1}^\infty$ y $\{\eta_k\}_{k=1}^\infty$ dos sucesiones tales que $\xi_k-\eta_k \kconv 0$. Para todo $k \in \N$ se tiene
    \begin{align*}
        |\widehat{f}(\xi_k) - \widehat{f}(\eta_k)| &=\frac{1}{\sqrt{2\pi}} \biggl|\int_\R f(x)(e^{i\xi_k x}-e^{i\eta_k x}) \, dx\biggr| \\
        &\leq \frac{1}{\sqrt{2\pi}} \int_\R |f(x)||e^{i\xi_k x} - e^{i\eta_kx}| \, dx \\
        &= \frac{1}{\sqrt{2\pi}} \int_\R |f(x)||e^{i\xi_kx}||1 - e^{i(\eta_k-\xi_k)x}| \, dx \\
        &= \frac{1}{\sqrt{2\pi}} \int_\R |f(x)||1 - e^{i(\eta_k-\xi_k)x}| \, dx.
    \end{align*}
    Veamos que
    \[\lim_{k \to \infty} \int_\R |f(x)||1 - e^{i(\eta_k-\xi_k)x}| = 0, \]
    lo que probará que $\widehat{f}$ es uniformemente continua. Como para todo $k \in \N$ y todo $x \in \R$ se tiene que
    \[|f(x)||1-e^{i(\eta_k-\xi_k)x}| \leq |f(x)|(1+|e^{i(\eta_k-\xi_k)x}|) = 2|f(x)|\]
    y $2f \in L^1(\R)$, por el teorema de la convergencia dominada,
    \[\lim_{k \to \infty} \int_\R |f(x)||1 - e^{i(\eta_k-\xi_k)x}|\, dx =\int_\R \lim_{k \to \infty} |f(x)||1 - e^{i(\eta_k-\xi_k)x}| \, dx = 0. \]
    \item Para todo $\xi \in \R$,
    \begin{align*}
        \widehat{f}(\xi) &= \frac{1}{\sqrt{2\pi}}\int_{-\infty}^0 -e^xe^{-i\xi x} \, dx+\frac{1}{\sqrt{2\pi}}\int_{0}^\infty e^{-x}e^{-i\xi x} \, dx \\
        &= \frac{1}{\sqrt{2\pi}}\lim_{n \to \infty}\left(\int_{0}^ne^{-(1+i\xi)x} \, dx - \int_{-n}^0 e^{(1-i\xi)x} \, dx\right) \\
        &= \frac{1}{\sqrt{2\pi}}\lim_{n \to \infty}\left(\left[-\frac{e^{-(1+i\xi)x}}{1+i\xi}\right]_{x=0}^{x=n} - \left[\frac{e^{(1-i\xi)x}}{1-i\xi}\right]_{x=-n}^{x=0}\right) \\
        &= \frac{1}{\sqrt{2\pi}}\left(\frac{1}{1+i\xi} -\frac{1}{1-i\xi}\right) = \frac{1-i\xi-(1+i\xi)}{\sqrt{2\pi}(1+i\xi)(1-i\xi)}  = -i\frac{2\xi}{\sqrt{2\pi}(1+\xi^2)}
    \end{align*}
    El teorema de inversión para la transformada de Fourier es el siguiente:
    \begin{theorem}
        Si $f \in L^1(\R)$ y $\widehat{f} \in L^1(\R)$, entonces
        \[f(x) = \frac{1}{\sqrt{2\pi}}\int_\R \widehat{f}(\xi)e^{i\xi x} \, dx\]
        para casi todo $x \in \R$.
    \end{theorem}
    Si fuese $\widehat{f}\in L^1(\R)$, el teorema de inversión permitiría afirmar que $f(x) = \widehat{\widehat{f}}(-x)$ en casi todo $x \in \R$, es decir, que $f$ es igual a una función continua en casi todo punto. Esto es imposible porque es no se puede cambiar el valor de $f$ en $0$ de manera que la nueva función obtenida sea continua en $0$. Por tanto, $\widehat{f} \not\in L^1(\R)$.
\end{enumerate}
\end{solution}

\begin{solution}
\hfill
\begin{enumerate}
    \item \textbf{Teorema.} \textit{Sea $\Omega \subset \R^n$ un dominio acotado, y sea $u \in \mathcal{C}^2(\Omega) \cap \mathcal{C}(\overline{\Omega})$ tal que $\Delta u(x) \geq 0$ para todo $x \in \Omega$. Entonces
    \[\max_{x \in \overline{\Omega}} u(x) = \max_{x \in \partial \Omega} u(x).\]
    Demostración.} Supóngase primero que $\Delta u(x) > 0$ para todo $x \in \Omega$. Si existiese $x_0 \in \Omega$ tal que
    \[u(x_0) = \max_{x \in \overline{\Omega}}u(x),\]
    entonces, por ser la matriz hessiana de $u$ en $x_0$ semidefinida negativa, se tendría que $\Delta u(x) \leq 0$, que contradice lo supuesto. Por tanto,
    \[\max_{x \in \overline{\Omega}} u(x) = \max_{x \in \partial\Omega} u(x).\]

    Supóngase ahora que $\Delta u(x) \geq 0$ para todo $x \in \Omega$. Sea $K >0$ y sea $v \colon \overline{\Omega} \to \R$ la función dada por
    \[v(x) = u(x) + K\|x\|^2 = u(x) + K(x_1^2+\mathellipsis+x_n^2).\]
    Es claro que $v \in \mathcal{C}^2(\Omega) \cap \mathcal{C}(\overline{\Omega})$, y además, para todo $x \in \Omega$,
    \[\Delta v(x) = \underbracket{\Delta u(x)}_{\geq 0} + \underbracket{2nK}_{>0} >0.\]
    Por lo probado anteriormente, se tiene que
    \[\max_{x \in \overline{\Omega}} v(x) = \max_{x \in \partial \Omega} v(x).\]
    Por otra parte, como $\overline{\Omega}$ es acotado, existe $R>0$ tal que $\|x\| \leq R$ para todo $x \in \Omega$. Como consecuencia de todo esto,
    \[\max_{x \in \overline{\Omega}} u(x) \leq \max_{x \in \overline{\Omega}} v(x) = \max_{x \in \partial \Omega} v(x) \leq \max_{x \in \partial \Omega} u(x) + KR^2, \tag{$\ast$}\]
    utilizándose en la primera desigualdad que $u(x) \leq v(x)$ para todo $x \in \overline{\Omega}$, y en la segunda desigualdad que $v(y) \leq \max_{x \in \partial \Omega} u(x)+KR^2$ para todo $y \in \partial\Omega$. Como $(\ast)$ es cierto para todo $K > 0$, tomando límites cuando $K \to 0^+$ se obtiene
    \[\max_{x \in \overline{\Omega}}u(x) \leq \max_{x \in \partial\Omega} u(x).\]
    Y como $\partial\Omega \subset \overline{\Omega}$, la otra desigualdad es trivial. \qed
    \begin{corollary}
        Sea $\Omega \subset \R^n$ y sea $\varphi \in \mathcal{C}(\partial \Omega)$. Entonces el problema
        \[\left\{\begin{alignedat}{2}
            \Delta u(x) &= 0, \qquad &x \in \Omega, \\
            u(x) &= \varphi(x), \qquad &x \in \partial \Omega,
        \end{alignedat}\right.\]
        tiene, a lo sumo, una solución.
    \end{corollary}
    \vspace{-\baselineskip}
    \begin{proof}
        Sean $u_1,u_2 \in \mathcal{C}^2(\Omega) \cap \mathcal{C}(\overline{\Omega})$ dos soluciones del problema anterior. Sea $u \equiv u_1 - u_2$ y veamos que $u \equiv 0$, lo que probará que el problema no tiene más que una solución. Claramente $u \mathcal{C}^2(\Omega) \cap \mathcal{C}(\overline{\Omega})$ y $\Delta u(x) = 0$ para todo $x \in \Omega$, luego, por el principio del máximo débil,
        \[\max_{x \in \overline{\Omega}} u(x) = \max_{x \in \partial \Omega} u(x).\]
        Además, $-u \mathcal{C}^2(\Omega) \cap \mathcal{C}(\overline{\Omega})$ y $\Delta (-u)(x) = 0$ para todo $x \in \Omega$, luego, por el principio del máximo débil,
        \[\max_{x \in \overline{\Omega}} -u(x) = \max_{x \in \partial \Omega} -u(x),\]
        es decir,
        \[\min_{x \in \overline{\Omega}} u(x) = \min_{x \in \partial \Omega} u(x).\]
        En consecuencia,
        \begin{align*}
            \max_{x \in \overline{\Omega}} |u(x)| &= \max\left\{\max_{x \in \overline{\Omega}}u(x),-\min_{x \in \overline{\Omega}}u(x)\right\} = \max\left\{\max_{x \in \partial\Omega}u(x),-\min_{x \in \partial\Omega}u(x)\right\} \\
            &= \max_{x \in \partial \Omega} |u(x)|.
        \end{align*}
        Pero $u(x) = u_1(x)-u_2(x)=\varphi(x)-\varphi(x)=0$ para todo $x \in \partial\Omega$, luego 
        \[\max_{x \in \partial \Omega} |u(x)| = 0,\]
        y por tanto,
        \[\max_{x \in \overline{\Omega}} |u(x)| = 0,\]
        concluyéndose que $u(x) = 0$ para todo $x \in \overline{\Omega}$.
    \end{proof}
    \item Supóngase que $u \not\equiv 0$ (si $u \equiv 0$, el resultado es trivial). Sea $x_0 \in \overline{D}$ tal que
    \[|u(x_0)| = \max_{x \in \overline{D}} |u(x)|,\]
    y veamos que 
    \[ |u(x_0)| =  \max_{x \in \partial D} |u(x)|.\]
    Como $\lim_{\|x\| \to \infty} u(x) = 0$, asociado a $|u(x_0)|$ (que es un número real positivo porque $u \not\equiv 0$) existe $R_0 > 0$ (que puede tomarse mayor que $1$) tal que para todo $x \in D$ con $\|x\| \geq R_0$ se tiene que $|u(x)| < |u(x_0)|$. Nótese que esto implica que $\|x_0\| < R_0$. Sea $\Omega = D \cap B(0,R_0)$, es decir, $\Omega$ es una corona circular de centro $0$, radio menor $1$ y radio mayor $R_0$. Claramente $\Omega$ es un dominio acotado. Como $\Delta u(x) = 0$ para todo $x \in \Omega \subset D$, razonando como en la demostración del corolario anterior se obtiene que
    \[\max_{x \in \overline{\Omega}} |u(x) | = \max_{x \in \partial{ \Omega}} |u(x)|.\]
    Por tanto,
    \[|u(x_0)|= \max_{x \in \overline{D}} |u(x)| = \max_{x \in \overline{\Omega}} |u(x)| =  \max_{x \in \partial{ \Omega}} |u(x)|,\]
    donde la segunda igualdad se verifica porque $x_0 \in \overline{\Omega}$ (pues $x_0 \in D$ y $\|x_0\| < R_0$) y $\overline{\Omega} \subset \overline{D}$. Como
    \[\partial\Omega = \{x \in \R^n \colon \|x\| = R_0\} \cup \{x \in \R^n \colon \|x\| = 1\} = \{x \in \R^n \colon \|x\| = R_0\} \cup \partial D\]
    y $|u(x_0)| > |u(x)|$ para todo $x \in D$ con $\|x\| = R_0$, entonces el máximo se alcanza en $\partial D$:
    \[|u(x_0)|= \max_{x \in \partial{ \Omega}} |u(x)| =  \max_{x \in \partial{ D}} |u(x)|.\]

\end{enumerate}
\end{solution}

\end{document}